%-------------------------------------------------------------------------
% GLFW Users Guide
% API Version: 2.4
% $Id: glfwug.tex,v 1.1 2003-11-03 21:23:21 marcus256 Exp $
%-------------------------------------------------------------------------

% Document class
\documentclass[a4paper,11pt,oneside]{report}

% Document title and API version
\newcommand{\glfwdoctype}[1][0]{Users Guide}
\newcommand{\glfwapiver}[1][0]{2.4}

% Common document settings and macros
%-------------------------------------------------------------------------
% Common document formatting and macros for GLFW manuals
% $Id: glfwdoc.sty,v 1.7 2003-11-14 21:26:45 marcus256 Exp $
%-------------------------------------------------------------------------

% Misc. document info
\date{\today}

% Packages
\usepackage{fancyhdr}
\usepackage{titling}
\usepackage{lastpage}
\usepackage{listings}
\usepackage{color}
\usepackage[overload]{textcase}
\usepackage{needspace}
\usepackage{times}

% Logo macros
\newcommand{\OpenGL}[1][0]{\textbf{OpenGL}\texttrademark}
\newcommand{\GLFW}[1][0]{\textbf{GLFW}}

% Encoding
\usepackage[latin1]{inputenc}
\usepackage[T1]{fontenc}

% Page formatting
\usepackage[hmargin=2.5cm]{geometry}
\raggedright
\raggedbottom
\sloppy
\usepackage{parskip}

% Header and footer
\pagestyle{fancy}
%\lhead{\textit{GLFW Reference Manual}}
\lhead{\textit{GLFW \glfwdoctype}}
\chead{API version \glfwapiver}
\rhead{Page \thepage/\pageref{LastPage}}
\lfoot{}
\cfoot{}
\rfoot{}
\renewcommand{\headrulewidth}{0.4pt}
\renewcommand{\footrulewidth}{0.0pt}

% Titlepage
\newcommand{\glfwmaketitle}{\begin{titlepage}\ \\%
                            \begin{center}%
                            \vspace{7.0cm}{\Huge\textbf{GLFW}}\\%
                            \rule{10.0cm}{0.5pt}\\%
                            \vspace{0.5cm}{\LARGE\textbf{\glfwdoctype}}\\%
                            \vspace{0.8cm}{\large\textbf{API version \glfwapiver}}\\%
                            \textit{\today}\\%
                            \vspace{1.5cm}\textbf{\textcopyright2002-2003 Marcus Geelnard}\\%
                            \end{center}\end{titlepage}\newpage}

% Colors
\definecolor{code}{rgb}{0.9,0.9,1.0}
\definecolor{link}{rgb}{0.6,0.0,0.0}

% Code listings
\lstset{frame=single,frameround=tttt,backgroundcolor=\color{code},%
        language=C,basicstyle={\ttfamily},%
        breaklines,breakindent=0pt,postbreak=\space\space\space\space}


% A simple hack for keeping lines together
\newenvironment{mysamepage}[1][2]{\begin{samepage}\needspace{#1\baselineskip}}{\end{samepage}}

% Macros for automating function reference entries
\newenvironment{refparameters}[1][0]{\begin{mysamepage}\textbf{Parameters}\\}{\end{mysamepage}\bigskip}
\newenvironment{refreturn}[1][0]{\begin{mysamepage}\textbf{Return values}\\}{\end{mysamepage}\bigskip}
\newenvironment{refdescription}[1][0]{\begin{mysamepage}\textbf{Description}\\}{\end{mysamepage}\bigskip}
\newenvironment{refnotes}[1][0]{\begin{mysamepage}\textbf{Notes}\\}{\end{mysamepage}\bigskip}

% hyperref (bookmarks, links etc) - use this package last
\usepackage[colorlinks=true,linkcolor=link,bookmarks=true,bookmarksopen=true,%
            pdfhighlight=/N,bookmarksnumbered=true,bookmarksopenlevel=1,%
            pdfview=FitH,pdfstartview=FitH]{hyperref}


% PDF specific document settings
\hypersetup{pdftitle={GLFW Users Guide}}
\hypersetup{pdfauthor={Marcus Geelnard}}
\hypersetup{pdfkeywords={GLFW,OpenGL,guide,manual}}


%-------------------------------------------------------------------------
% Document body
%-------------------------------------------------------------------------

\begin{document}

% Title page
\glfwmaketitle

% Summary, trademarks and table of contents
\pagenumbering{roman}
\setcounter{page}{1}

%-------------------------------------------------------------------------
% Summary and Trademarks
%-------------------------------------------------------------------------
\chapter*{Summary}

This document is a users guide for the \GLFW\ API. For a detailed
description of the \GLFW\ API you should refer to the
\textit{GLFW Reference Manual}.
\vspace{10cm}

\large
Trademarks

\small
OpenGL and IRIX are registered trademarks of Silicon Graphics, Inc.\linebreak
Microsoft, Windows and MS�-DOS are registered trademarks of Microsoft Corporation.\linebreak
Mac OS is a registered trademark of Apple Computer, Inc.\linebreak
Linux is a registered trademark of Linus Torvalds.\linebreak
FreeBSD is a registered trademark of Wind River Systems, Inc.\linebreak
Solaris is a trademark of Sun Microsystems, Inc.\linebreak
UNIX is a registered trademark of The Open Group.\linebreak
X Window System is a trademark of The Open Group.\linebreak
POSIX is a trademark of IEEE.\linebreak
Truevision, TARGA and TGA are registered trademarks of Truevision, Inc.\linebreak

All other trademarks mentioned in this document are the property of their respective owners.
\normalsize


%-------------------------------------------------------------------------
% Table of contents
%-------------------------------------------------------------------------
\tableofcontents
\pagebreak

% Document chapters starts here...
\pagenumbering{arabic}
\setcounter{page}{1}


%-------------------------------------------------------------------------
% Introduction
%-------------------------------------------------------------------------
\chapter{Introduction}
\thispagestyle{fancy}

\GLFW\ is a portable API (Application Program Interface) that handles
operating system specific tasks related to \OpenGL\ programming. While
\OpenGL\ in general is portable, easy to use and often results in tidy and
compact code, the operating system specific mechanisms that are required
to set up and manage an \OpenGL\ window are quite the opposite. \GLFW\ tries
to remedy this by providing the following functionality:

\begin{itemize}
\item Opening and managing an \OpenGL\ window.
\item Keyboard, mouse and joystick input.
\item A high precision timer.
\item Multi threading support.
\item Support for querying and using \OpenGL\ extensions.
\item Image file loading support.
\end{itemize}
\vspace{18pt}

All this functionality is implemented as a set of easy-to-use functions,
which makes it possible to write an \OpenGL\ application framework in just a
few lines of code. The \GLFW\ API is completely operating system and
platform independent, which makes it very simple to port \GLFW\ based \OpenGL\
applications to a variety of platforms.

Currently supported platforms are:
\begin{itemize}
\item Microsoft Windows\textsuperscript{\textregistered} 95/98/ME/NT/2000/XP/.NET Server.
\item Unix\textsuperscript{\textregistered} or Unix�-like systems running the
X Window System\texttrademark, e.g. Linux\textsuperscript{\textregistered},
IRIX\textsuperscript{\textregistered}, FreeBSD\textsuperscript{\textregistered},
Solaris\texttrademark, QNX\textsuperscript{\textregistered} and
Mac OS\textsuperscript{\textregistered} X.
\item Mac OS\textsuperscript{\textregistered} X (Carbon)\footnote{Only a subset of the \GLFW\ API is supported for this platform at the time of writing.}
\item AmigaOS\footnotemark[\value{footnote}]
\end{itemize}



\end{document}
