%-------------------------------------------------------------------------
% GLFW Users Guide
% API Version: 2.4
% $Id: glfwug.tex,v 1.2 2003-11-14 21:27:16 marcus256 Exp $
%-------------------------------------------------------------------------

% Document class
\documentclass[a4paper,11pt,oneside]{report}

% Document title and API version
\newcommand{\glfwdoctype}[1][0]{Users Guide}
\newcommand{\glfwapiver}[1][0]{2.4}

% Common document settings and macros
%-------------------------------------------------------------------------
% Common document formatting and macros for GLFW manuals
% $Id: glfwdoc.sty,v 1.7 2003-11-14 21:26:45 marcus256 Exp $
%-------------------------------------------------------------------------

% Misc. document info
\date{\today}

% Packages
\usepackage{fancyhdr}
\usepackage{titling}
\usepackage{lastpage}
\usepackage{listings}
\usepackage{color}
\usepackage[overload]{textcase}
\usepackage{needspace}
\usepackage{times}

% Logo macros
\newcommand{\OpenGL}[1][0]{\textbf{OpenGL}\texttrademark}
\newcommand{\GLFW}[1][0]{\textbf{GLFW}}

% Encoding
\usepackage[latin1]{inputenc}
\usepackage[T1]{fontenc}

% Page formatting
\usepackage[hmargin=2.5cm]{geometry}
\raggedright
\raggedbottom
\sloppy
\usepackage{parskip}

% Header and footer
\pagestyle{fancy}
%\lhead{\textit{GLFW Reference Manual}}
\lhead{\textit{GLFW \glfwdoctype}}
\chead{API version \glfwapiver}
\rhead{Page \thepage/\pageref{LastPage}}
\lfoot{}
\cfoot{}
\rfoot{}
\renewcommand{\headrulewidth}{0.4pt}
\renewcommand{\footrulewidth}{0.0pt}

% Titlepage
\newcommand{\glfwmaketitle}{\begin{titlepage}\ \\%
                            \begin{center}%
                            \vspace{7.0cm}{\Huge\textbf{GLFW}}\\%
                            \rule{10.0cm}{0.5pt}\\%
                            \vspace{0.5cm}{\LARGE\textbf{\glfwdoctype}}\\%
                            \vspace{0.8cm}{\large\textbf{API version \glfwapiver}}\\%
                            \textit{\today}\\%
                            \vspace{1.5cm}\textbf{\textcopyright2002-2003 Marcus Geelnard}\\%
                            \end{center}\end{titlepage}\newpage}

% Colors
\definecolor{code}{rgb}{0.9,0.9,1.0}
\definecolor{link}{rgb}{0.6,0.0,0.0}

% Code listings
\lstset{frame=single,frameround=tttt,backgroundcolor=\color{code},%
        language=C,basicstyle={\ttfamily},%
        breaklines,breakindent=0pt,postbreak=\space\space\space\space}


% A simple hack for keeping lines together
\newenvironment{mysamepage}[1][2]{\begin{samepage}\needspace{#1\baselineskip}}{\end{samepage}}

% Macros for automating function reference entries
\newenvironment{refparameters}[1][0]{\begin{mysamepage}\textbf{Parameters}\\}{\end{mysamepage}\bigskip}
\newenvironment{refreturn}[1][0]{\begin{mysamepage}\textbf{Return values}\\}{\end{mysamepage}\bigskip}
\newenvironment{refdescription}[1][0]{\begin{mysamepage}\textbf{Description}\\}{\end{mysamepage}\bigskip}
\newenvironment{refnotes}[1][0]{\begin{mysamepage}\textbf{Notes}\\}{\end{mysamepage}\bigskip}

% hyperref (bookmarks, links etc) - use this package last
\usepackage[colorlinks=true,linkcolor=link,bookmarks=true,bookmarksopen=true,%
            pdfhighlight=/N,bookmarksnumbered=true,bookmarksopenlevel=1,%
            pdfview=FitH,pdfstartview=FitH]{hyperref}


% PDF specific document settings
\hypersetup{pdftitle={GLFW Users Guide}}
\hypersetup{pdfauthor={Marcus Geelnard}}
\hypersetup{pdfkeywords={GLFW,OpenGL,guide,manual}}


%-------------------------------------------------------------------------
% Document body
%-------------------------------------------------------------------------

\begin{document}

% Title page
\glfwmaketitle

% Summary, trademarks and table of contents
\pagenumbering{roman}
\setcounter{page}{1}

%-------------------------------------------------------------------------
% Summary and Trademarks
%-------------------------------------------------------------------------
\chapter*{Summary}

This document is a users guide for the \GLFW\ API. For a detailed
description of the \GLFW\ API you should refer to the
\textit{GLFW Reference Manual}.
\vspace{10cm}

\large
Trademarks

\small
OpenGL and IRIX are registered trademarks of Silicon Graphics, Inc.\linebreak
Microsoft, Windows and MS�-DOS are registered trademarks of Microsoft Corporation.\linebreak
Mac OS is a registered trademark of Apple Computer, Inc.\linebreak
Linux is a registered trademark of Linus Torvalds.\linebreak
FreeBSD is a registered trademark of Wind River Systems, Inc.\linebreak
Solaris is a trademark of Sun Microsystems, Inc.\linebreak
UNIX is a registered trademark of The Open Group.\linebreak
X Window System is a trademark of The Open Group.\linebreak
POSIX is a trademark of IEEE.\linebreak
Truevision, TARGA and TGA are registered trademarks of Truevision, Inc.\linebreak

All other trademarks mentioned in this document are the property of their respective owners.
\normalsize


%-------------------------------------------------------------------------
% Table of contents
%-------------------------------------------------------------------------
\tableofcontents
\pagebreak

% Document chapters starts here...
\pagenumbering{arabic}
\setcounter{page}{1}


%-------------------------------------------------------------------------
% Introduction
%-------------------------------------------------------------------------
\chapter{Introduction}
\thispagestyle{fancy}
\GLFW\ is a portable API (Application Program Interface) that handles
operating system specific tasks related to \OpenGL\ programming. While
\OpenGL\ in general is portable, easy to use and often results in tidy and
compact code, the operating system specific mechanisms that are required
to set up and manage an \OpenGL\ window are quite the opposite. \GLFW\ tries
to remedy this by providing the following functionality:

\begin{itemize}
\item Opening and managing an \OpenGL\ window.
\item Keyboard, mouse and joystick input.
\item A high precision timer.
\item Multi threading support.
\item Support for querying and using \OpenGL\ extensions.
\item Image file loading support.
\end{itemize}
\vspace{18pt}

All this functionality is implemented as a set of easy-to-use functions,
which makes it possible to write an \OpenGL\ application framework in just a
few lines of code. The \GLFW\ API is completely operating system and
platform independent, which makes it very simple to port \GLFW\ based \OpenGL\
applications to a variety of platforms.

Currently supported platforms are:
\begin{itemize}
\item Microsoft Windows\textsuperscript{\textregistered} 95/98/ME/NT/2000/XP/.NET Server.
\item Unix\textsuperscript{\textregistered} or Unix�-like systems running the
X Window System\texttrademark, e.g. Linux\textsuperscript{\textregistered},
IRIX\textsuperscript{\textregistered}, FreeBSD\textsuperscript{\textregistered},
Solaris\texttrademark, QNX\textsuperscript{\textregistered} and
Mac OS\textsuperscript{\textregistered} X.
\item Mac OS\textsuperscript{\textregistered} X (Carbon)\footnote{Only a subset of the \GLFW\ API is supported for this platform at the time of writing.}
\item AmigaOS\footnotemark[\value{footnote}]
\end{itemize}


%-------------------------------------------------------------------------
% Getting Started
%-------------------------------------------------------------------------
\chapter{Getting Started}
\thispagestyle{fancy}
In this chapter you will learn how to write a simple \OpenGL\ application
using \GLFW . We start by initializing \GLFW , then we open a window and
read some user keyboard input.


\section{Initializing GLFW}
Before using any of the \GLFW\ functions, it is necessary to call
\textbf{glfwInit}. It initializes internal working variables that are used
by other \GLFW\ functions. The C syntax is:

\begin{lstlisting}
int glfwInit( void )
\end{lstlisting}

\textbf{glfwInit} returns GL\_TRUE if initialization succeeded, or
GL\_FALSE if it failed.

When your application is done using \GLFW , typically at the very end of
the program, you should call \textbf{glfwTerminate}, which makes a clean
up and places \GLFW\ in a non-initialized state (i.e. it is necessary to
call \textbf{glfwInit} again before using any \GLFW\ functions). The C
syntax is:

\begin{lstlisting}
void glfwTerminate( void )
\end{lstlisting}

Among other things, \textbf{glfwTerminate} closes the \OpenGL\ window
unless it was closed manually, and kills any running threads that were
created using \GLFW .


\section{Opening An OpenGL Window}
Opening an \OpenGL\ window is done with the function
\textbf{glfwOpenWindow}. The function takes nine arguments, which are used
to describe the following properties of the window to open:

\begin{itemize}
\item Window dimensions (width and height) in pixels.
\item Color and alpha buffer depth.
\item Depth buffer (Z-buffer) depth.
\item Stencil buffer depth.
\item Fullscreen or windowed mode.
\end{itemize}

The C language syntax for \textbf{glfwOpenWindow} is:
\begin{lstlisting}
int glfwOpenWindow( int width, int height,
    int redbits, int greenbits, int bluebits,
    int alphabits, int depthbits, int stencilbits,
    int mode )
\end{lstlisting}

\textbf{glfwOpenWindow} returns GL\_TRUE if the window was opened
correctly, or GL\_FALSE if \GLFW\ failed to open the window.

\GLFW\ tries to open a window that best matches the requested parameters.
Some parameters may be omitted by setting them to zero, which will result
in \GLFW\ either using a default value, or the related functionality to be
disabled. For instance, if \textit{width} and \textit{height} are both
zero, \GLFW\ will use a window resolution of 640x480. If
\textit{depthbits} is zero, the opened window may not have a depth buffer.

The \textit{mode} argument is used to specify if the window is to be a
s.c. fullscreen window, or a regular window.

If \textit{mode} is GLFW\_FULLSCREEN, the window will cover the entire
screen and no window borders will be visible. If possible, the video mode
will be changed to the mode that closest matches the \textit{width},
\textit{height}, \textit{redbits}, \textit{greenbits}, \textit{bluebits}
and \textit{alphabits} arguments. Furthermore, the mouse pointer will be
hidden, and screensavers are prohibited. This is usually the best mode for
games and demos.

If \textit{mode} is GLFW\_WINDOW, the window will be opened as a normal
window on the desktop. The mouse pointer will not be hidden, and
screensavers are allowed to be activated.

To close the window, you can either use \textbf{glfwTerminate}, as
described earlier, or you can use the more explicit approach by calling
\textbf{glfwCloseWindow}, which has the C syntax:

\begin{lstlisting}
void glfwCloseWindow( void )
\end{lstlisting}


\section{Using Keyboard Input}
\GLFW\ provides several means for receiving user input, which will be
discussed in more detail later on in this manual. One of the simplest ways
of checking for keyboard input is to use the function \textbf{glfwGetKey}:

\begin{lstlisting}
int glfwGetKey( int key )
\end{lstlisting}

It queries the current status of individual keyboard keys. The argument
\textit{key} specifies which key to check, and it can be either an
uppercase printable ISO 8859-1 (Latin 1) character (e.g. `A', `3' or `.'),
or a special key identifier (see the \textit{GLFW Reference Manual} for a
list of special key identifiers). \textbf{glfwGetKey} returns GLFW\_PRESS
(or 1) if the key is currently held down, or GLFW\_RELEASE (or 0) if the
key is not being held down. For example:

\begin{lstlisting}
A_pressed = glfwGetKey( 'A' );
esc_pressed = glfwGetKey( GLFW_KEY_ESC );
\end{lstlisting}

In order for \textbf{glfwGetKey} to have any effect, you need to poll for
input events on a regular basis. This can be done in one of two ways:

\begin{enumerate}
\item Implicitly by calling \textbf{glfwSwapBuffers} often.
\item Explicitly by calling \textbf{glfwPollEvents} often.
\end{enumerate}

In general you do not have to care about this, since you will normally
call \textbf{glfwSwapBuffers} to swap front and back rendering buffers
every animation frame anyway. If, however, this is not the case, you
should call \textbf{glfwPollEvents} in the order of 10-100 times per
second in order for \GLFW\ to maintain an up to date input state.


\section{Putting It Together: A Minimal GLFW Application}
Now that you know how to initialize \GLFW , open a window and poll for
keyboard input, let us exemplify this with a simple \OpenGL\ program. In
the following example some error-checking has been omitted for the sake of
brevity:

\begin{lstlisting}
#include <GL/glfw.h>

int main( void )
{
    int running = GL_TRUE;

    // Initialize GLFW
    glfwInit();

    // Open an OpenGL window
    if( !glfwOpenWindow( 300,300, 0,0,0,0,0,0, GLFW_WINDOW ) )
    {
        glfwTerminate();
        return 0;
    }

    // Main loop
    while( running )
    {
        // OpenGL rendering goes here...
        glClear( GL_COLOR_BUFFER_BIT );

        // Swap front and back rendering buffers
        glfwSwapBuffers();

        // Check if ESC key was pressed or window was closed
        running = !glfwGetKey( GLFW_KEY_ESC ) &&
                  glfwGetWindowParam( GLFW_OPENED );
    }

    // Close window and terminate GLFW
    glfwTerminate();

    // Exit program
    return 0;
}
\end{lstlisting}

The program opens a 300x300 window and runs in a loop until the escape key
is pressed, or the window was closed. All the OpenGL ``rendering'' that is
done in this example is to clear the window, using the \textbf{glClear}
function.



%-------------------------------------------------------------------------
% Index
%-------------------------------------------------------------------------
% ...

\end{document}
