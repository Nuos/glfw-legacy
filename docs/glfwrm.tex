%-------------------------------------------------------------------------
% GLFW Reference Manual
% API Version: 2.4
% $Id: glfwrm.tex,v 1.13 2004-01-31 20:15:15 marcus256 Exp $
%-------------------------------------------------------------------------

% Document class
\documentclass[a4paper,11pt,oneside]{report}

% Document title and API version
\newcommand{\glfwdoctype}[1][0]{Reference Manual}
\newcommand{\glfwapiver}[1][0]{2.4}

% Common document settings and macros
%-------------------------------------------------------------------------
% Common document formatting and macros for GLFW manuals
% $Id: glfwdoc.sty,v 1.7 2003-11-14 21:26:45 marcus256 Exp $
%-------------------------------------------------------------------------

% Misc. document info
\date{\today}

% Packages
\usepackage{fancyhdr}
\usepackage{titling}
\usepackage{lastpage}
\usepackage{listings}
\usepackage{color}
\usepackage[overload]{textcase}
\usepackage{needspace}
\usepackage{times}

% Logo macros
\newcommand{\OpenGL}[1][0]{\textbf{OpenGL}\texttrademark}
\newcommand{\GLFW}[1][0]{\textbf{GLFW}}

% Encoding
\usepackage[latin1]{inputenc}
\usepackage[T1]{fontenc}

% Page formatting
\usepackage[hmargin=2.5cm]{geometry}
\raggedright
\raggedbottom
\sloppy
\usepackage{parskip}

% Header and footer
\pagestyle{fancy}
%\lhead{\textit{GLFW Reference Manual}}
\lhead{\textit{GLFW \glfwdoctype}}
\chead{API version \glfwapiver}
\rhead{Page \thepage/\pageref{LastPage}}
\lfoot{}
\cfoot{}
\rfoot{}
\renewcommand{\headrulewidth}{0.4pt}
\renewcommand{\footrulewidth}{0.0pt}

% Titlepage
\newcommand{\glfwmaketitle}{\begin{titlepage}\ \\%
                            \begin{center}%
                            \vspace{7.0cm}{\Huge\textbf{GLFW}}\\%
                            \rule{10.0cm}{0.5pt}\\%
                            \vspace{0.5cm}{\LARGE\textbf{\glfwdoctype}}\\%
                            \vspace{0.8cm}{\large\textbf{API version \glfwapiver}}\\%
                            \textit{\today}\\%
                            \vspace{1.5cm}\textbf{\textcopyright2002-2003 Marcus Geelnard}\\%
                            \end{center}\end{titlepage}\newpage}

% Colors
\definecolor{code}{rgb}{0.9,0.9,1.0}
\definecolor{link}{rgb}{0.6,0.0,0.0}

% Code listings
\lstset{frame=single,frameround=tttt,backgroundcolor=\color{code},%
        language=C,basicstyle={\ttfamily},%
        breaklines,breakindent=0pt,postbreak=\space\space\space\space}


% A simple hack for keeping lines together
\newenvironment{mysamepage}[1][2]{\begin{samepage}\needspace{#1\baselineskip}}{\end{samepage}}

% Macros for automating function reference entries
\newenvironment{refparameters}[1][0]{\begin{mysamepage}\textbf{Parameters}\\}{\end{mysamepage}\bigskip}
\newenvironment{refreturn}[1][0]{\begin{mysamepage}\textbf{Return values}\\}{\end{mysamepage}\bigskip}
\newenvironment{refdescription}[1][0]{\begin{mysamepage}\textbf{Description}\\}{\end{mysamepage}\bigskip}
\newenvironment{refnotes}[1][0]{\begin{mysamepage}\textbf{Notes}\\}{\end{mysamepage}\bigskip}

% hyperref (bookmarks, links etc) - use this package last
\usepackage[colorlinks=true,linkcolor=link,bookmarks=true,bookmarksopen=true,%
            pdfhighlight=/N,bookmarksnumbered=true,bookmarksopenlevel=1,%
            pdfview=FitH,pdfstartview=FitH]{hyperref}


% PDF specific document settings
\hypersetup{pdftitle={GLFW Reference Manual}}
\hypersetup{pdfauthor={Marcus Geelnard}}
\hypersetup{pdfkeywords={GLFW,OpenGL,reference,manual}}


%-------------------------------------------------------------------------
% Document body
%-------------------------------------------------------------------------

\begin{document}

\pagestyle{plain}

% Title page
\glfwmaketitle

% Summary, trademarks and table of contents
\pagenumbering{roman}
\setcounter{page}{1}

%-------------------------------------------------------------------------
% Summary and Trademarks
%-------------------------------------------------------------------------
\chapter*{Summary}

This document is a function reference manual for the \GLFW\ API. For a
description of how to use \GLFW\ you should refer to the \textit{GLFW Users Guide}.
\vspace{10cm}

\large
Trademarks

\small
OpenGL and IRIX are registered trademarks of Silicon Graphics, Inc.\linebreak
Microsoft, Windows and MS�-DOS are registered trademarks of Microsoft Corporation.\linebreak
Mac OS is a registered trademark of Apple Computer, Inc.\linebreak
Linux is a registered trademark of Linus Torvalds.\linebreak
FreeBSD is a registered trademark of Wind River Systems, Inc.\linebreak
Solaris is a trademark of Sun Microsystems, Inc.\linebreak
UNIX is a registered trademark of The Open Group.\linebreak
X Window System is a trademark of The Open Group.\linebreak
POSIX is a trademark of IEEE.\linebreak
Truevision, TARGA and TGA are registered trademarks of Truevision, Inc.\linebreak

All other trademarks mentioned in this document are the property of their respective owners.
\normalsize


%-------------------------------------------------------------------------
% Table of contents
%-------------------------------------------------------------------------
\tableofcontents

%-------------------------------------------------------------------------
% List of tables
%-------------------------------------------------------------------------
\listoftables
\pagebreak


% Document chapters starts here...
\pagenumbering{arabic}
\setcounter{page}{1}

\pagestyle{fancy}


%-------------------------------------------------------------------------
% Introduction
%-------------------------------------------------------------------------
\chapter{Introduction}
\thispagestyle{fancy}

\GLFW\ is a portable API (Application Program Interface) that handles
operating system specific tasks related to \OpenGL\ programming. While
\OpenGL\ in general is portable, easy to use and often results in tidy and
compact code, the operating system specific mechanisms that are required
to set up and manage an \OpenGL\ window are quite the opposite. \GLFW\ tries
to remedy this by providing the following functionality:

\begin{itemize}
\item Opening and managing an \OpenGL\ window.
\item Keyboard, mouse and joystick input.
\item A high precision timer.
\item Multi threading support.
\item Support for querying and using \OpenGL\ extensions.
\item Image file loading support.
\end{itemize}
\vspace{18pt}

All this functionality is implemented as a set of easy-to-use functions,
which makes it possible to write an \OpenGL\ application framework in just a
few lines of code. The \GLFW\ API is completely operating system and
platform independent, which makes it very simple to port \GLFW\ based \OpenGL\
applications to a variety of platforms.

Currently supported platforms are:
\begin{itemize}
\item Microsoft Windows\textsuperscript{\textregistered} 95/98/ME/NT/2000/XP/.NET Server.
\item Unix\textsuperscript{\textregistered} or Unix�-like systems running the
X Window System\texttrademark, e.g. Linux\textsuperscript{\textregistered},
IRIX\textsuperscript{\textregistered}, FreeBSD\textsuperscript{\textregistered},
Solaris\texttrademark, QNX\textsuperscript{\textregistered} and
Mac OS\textsuperscript{\textregistered} X.
\item Mac OS\textsuperscript{\textregistered} X (Carbon)\footnote{Only a subset of the \GLFW\ API is supported for this platform at the time of writing.}
\item AmigaOS\footnotemark[\value{footnote}]
\end{itemize}



%-------------------------------------------------------------------------
% Function Reference
%-------------------------------------------------------------------------
\chapter{Function Reference}
\thispagestyle{fancy}

%-------------------------------------------------------------------------
\section{GLFW Initialization and Termination}
Before any \GLFW\ functions can be used, \GLFW\ must be initialized to
ensure proper functionality, and before a program terminates, \GLFW\ has to
be terminated in order to free up resources etc.


%-------------------------------------------------------------------------
\subsection{glfwInit}

\textbf{C language syntax}
\begin{lstlisting}
int glfwInit( void )
\end{lstlisting}

\begin{refparameters}
none
\end{refparameters}

\begin{refreturn}
If the function succeeds, GL\_TRUE is returned.\\
If the function fails, GL\_FALSE is returned.
\end{refreturn}

\begin{refdescription}
The glfwInit function initializes \GLFW. No other \GLFW\ functions may be
used before this function has been called.
\end{refdescription}

\begin{refnotes}
This function may take several seconds to complete on some systems, while
on other systems it may take only a fraction of a second to complete.
\end{refnotes}


%-------------------------------------------------------------------------
\subsection{glfwTerminate}

\textbf{C language syntax}
\begin{lstlisting}
void glfwTerminate( void )
\end{lstlisting}

\begin{refparameters}
none
\end{refparameters}

\begin{refreturn}
none
\end{refreturn}

\begin{refdescription}
The function terminates \GLFW. Among other things it closes the window,
if it is opened, and kills any running threads. This function must be
called before a program exits.
\end{refdescription}


%-------------------------------------------------------------------------
\subsection{glfwGetVersion}

\textbf{C language syntax}
\begin{lstlisting}
void glfwGetVersion( int *major, int *minor, int *rev )
\end{lstlisting}

\begin{refparameters}
\begin{description}
\item [\textit{major}]\ \\
  Pointer to an integer that will hold the major version number.
\item [\textit{minor}]\ \\
  Pointer to an integer that will hold the minor version number.
\item [\textit{rev}]\ \\
  Pointer to an integer that will hold the revision.
\end{description}
\end{refparameters}

\begin{refreturn}
The function returns the major and minor version numbers and the revision
for the currently linked \GLFW\ library.
\end{refreturn}

\begin{refdescription}
The function returns the \GLFW\ library version.
\end{refdescription}


%-------------------------------------------------------------------------
\pagebreak
\section{Window Handling}
The main functionality of \GLFW\ is to provide a simple interface to
\OpenGL\ window management. \GLFW\ can open one window, which can be
either a normal desktop window or a fullscreen window.


%-------------------------------------------------------------------------
\subsection{glfwOpenWindow}

\textbf{C language syntax}
\begin{lstlisting}
int glfwOpenWindow( int width, int height, int redbits,
    int greenbits, int bluebits, int alphabits, int depthbits,
    int stencilbits, int mode )
\end{lstlisting}

\begin{refparameters}
\begin{description}
\item [\textit{width}]\ \\
  The width of the window. If \textit{width} is zero, it will be
  calculated as ${width=\frac{4}{3}height}$, if \textit{height} is not
  zero. If both \textit{width} and \textit{height} are zero, then
  \textit{width} will be set to 640.
\item [\textit{hieght}]\ \\
  The height of the window. If \textit{height} is zero, it will be
  calculated as ${height=\frac{3}{4}width}$, if \textit{width} is not
  zero. If both \textit{width} and \textit{height} are zero, then
  \textit{height} will be set to 480.
\item [\textit{redbits, greenbits, bluebits}]\ \\
  The number of bits to use for each color component of the color buffer
  (0 means default color depth). For instance, setting \textit{redbits=5,
  greenbits=6, and bluebits=5} will generate a 16-�bit color buffer, if
  possible.
\item [\textit{alphabits}]\ \\
  The number of bits to use for the alpha buffer (0 means no alpha
  buffer).
\item [\textit{depthbits}]\ \\
  The number of bits to use for the depth buffer (0 means no depth
  buffer).
\item [\textit{stencilbits}]\ \\
  The number of bits to use for the stencil buffer (0 means no stencil
  buffer).
\item [\textit{mode}]\ \\
  Selects which type of \OpenGL\ window to use. \textit{mode} can be
  either GLFW\_WINDOW, which will generate a normal desktop window, or
  GLFW\_FULLSCREEN, which will generate a window which covers the entire
  screen. When GLFW\_FULLSCREEN is selected, the video mode will be
  changed to the resolution that closest matches the \textit{width} and
  \textit{height} parameters.
\end{description}
\end{refparameters}

\begin{refreturn}
If the function succeeds, GL\_TRUE is returned.\\
If the function fails, GL\_FALSE is returned.
\end{refreturn}

\begin{refdescription}
The function opens a window that best matches the parameters given to the
function. How well the resulting window matches the desired window depends
mostly on the available hardware and \OpenGL\ drivers. In general,
selecting a fullscreen mode has better chances of generating a close match
than does a normal desktop window, since \GLFW\ can freely select from all
the available video modes. A desktop window is normally restricted to the
video mode of the desktop.
\end{refdescription}

\begin{refnotes}
For additional control of window properties, see
\textbf{glfwOpenWindowHint}.

In fullscreen mode the mouse cursor is hidden by default, and any system
screensavers are prohibited from starting. In windowed mode the mouse
cursor is visible, and screensavers are allowed to start. To change the
visibility of the mouse cursor, use \textbf{glfwEnable} or
\textbf{glfwDisable} with the argument GLFW\_MOUSE\_CURSOR.

In order to determine the actual properties of an opened window, use
\textbf{glfwGetWindowParam} and \textbf{glfwGetWindowSize} (or
\textbf{glfwSetWindowSizeCallback}).
\end{refnotes}


%-------------------------------------------------------------------------
\begin{table}[p]
\begin{center}
\begin{tabular}{|l|l|p{7.0cm}|} \hline \raggedright
\textbf{Name}            & \textbf{Default} & \textbf{Description} \\ \hline
GLFW\_REFRESH\_RATE      & 0                & Vertical monitor refresh rate in Hz (only used for fullscreen windows). Zero means system default.\\ \hline
GLFW\_ACCUM\_RED\_BITS   & 0                & Number of bits for the red channel of the accumulator buffer.\\ \hline
GLFW\_ACCUM\_GREEN\_BITS & 0                & Number of bits for the green channel of the accumulator buffer.\\ \hline
GLFW\_ACCUM\_BLUE\_BITS  & 0                & Number of bits for the blue channel of the accumulator buffer.\\ \hline
GLFW\_ACCUM\_ALPHA\_BITS & 0                & Number of bits for the alpha channel of the accumulator buffer.\\ \hline
GLFW\_AUX\_BUFFERS       & 0                & Number of auxiliary buffers.\\ \hline
GLFW\_STEREO             & GL\_FALSE        & Specify if stereo rendering should be supported (can be GL\_TRUE or GL\_FALSE).\\ \hline
\end{tabular}
\end{center}
\caption{Targets for \textbf{glfwOpenWindowHint}}
\label{tab:winhints}
\end{table}


%-------------------------------------------------------------------------
\subsection{glfwOpenWindowHint}

\textbf{C language syntax}
\begin{lstlisting}
void glfwOpenWindowHint( int target, int hint )
\end{lstlisting}

\begin{refparameters}
\begin{description}
\item [\textit{target}]\ \\
  Can be any of the constants in the table \ref{tab:winhints}.
\item [\textit{hint}]\ \\
  An integer giving the value of the corresponding target (see table
  \ref{tab:winhints}).
\end{description}
\end{refparameters}

\begin{refreturn}
none
\end{refreturn}

\begin{refdescription}
The function sets additional properties for a window that is to be opened.
For a hint to be registered, the function must be called before calling
\textbf{glfwOpenWindow}. When the \textbf{glfwOpenWindow} function is
called, any hints that were registered with the \textbf{glfwOpenWindowHint}
function are used for setting the corresponding window properties, and
then all hints are reset to their default values.
\end{refdescription}

\begin{refnotes}
In order to determine the actual properties of an opened window, use
\textbf{glfwGetWindowParam} (after the window has been opened).

GLFW\_STEREO is a hard constraint. If stereo rendering is requested, but
no stereo rendering capable pixel formats / visuals are available,
\textbf{glfwOpenWindow} will fail.

GLFW\_REFRESH\_RATE is only supported under Windows.

The GLFW\_REFRESH\_RATE property should be used with caution. Most
systems have default values for monitor refresh rates that are optimal
for the specific system. Specifying the refresh rate can override these
settings, which can result in suboptimal operation. The monitor may be
unable to display the resulting video signal, or in the worst case it may
even be damaged!
\end{refnotes}


%-------------------------------------------------------------------------
\subsection{glfwCloseWindow}

\textbf{C language syntax}
\begin{lstlisting}
void glfwCloseWindow( void )
\end{lstlisting}

\begin{refparameters}
none
\end{refparameters}

\begin{refreturn}
none
\end{refreturn}

\begin{refdescription}
The function closes an opened window and destroys the associated \OpenGL\
context.
\end{refdescription}


%-------------------------------------------------------------------------
\subsection{glfwSetWindowTitle}

\textbf{C language syntax}
\begin{lstlisting}
void glfwSetWindowTitle( const char *title )
\end{lstlisting}

\begin{refparameters}
\begin{description}
\item [\textit{title}]\ \\
  Pointer to a  null terminated ISO~8859-1 (8-bit Latin~1) string that
  holds the title of the window.
\end{description}
\end{refparameters}

\begin{refreturn}
none
\end{refreturn}

\begin{refdescription}
The function changes the title of the opened window.
\end{refdescription}

\begin{refnotes}
The title property of a window is often used in situations other than for
the window title, such as the title of an application icon when it is in
iconified state.
\end{refnotes}


%-------------------------------------------------------------------------
\subsection{glfwSetWindowSize}

\textbf{C language syntax}
\begin{lstlisting}
void glfwSetWindowSize( int width, int height )
\end{lstlisting}

\begin{refparameters}
\begin{description}
\item [\textit{width}]\ \\
  Width of the window.
\item [\textit{height}]\ \\
  Height of the window.
\end{description}
\end{refparameters}

\begin{refreturn}
none
\end{refreturn}

\begin{refdescription}
The function changes the size of an opened window. The \textit{width} and
\textit{height} parameters denote the size of the client area of the
window (i.e. excluding any window borders and decorations).

If the window is in fullscreen mode, the video mode will be changed to a
resolution that closest matches the width and height parameters (the
number of color bits will not be changed).
\end{refdescription}

\begin{refnotes}
The \OpenGL\ context is guaranteed to be preserved after calling
\textbf{glfwSetWindowSize}, even if the video mode is changed.

Changing the size of a fullscreen window is not supported under AmigaOS or
DOS, since that would destroy the associated \OpenGL\ context.
\end{refnotes}


%-------------------------------------------------------------------------
\subsection{glfwSetWindowPos}

\textbf{C language syntax}
\begin{lstlisting}
void glfwSetWindowPos( int x, int y )
\end{lstlisting}

\begin{refparameters}
\begin{description}
\item [\textit{x}]\ \\
  Horizontal position of the window, relative to the upper left corner
  of the desktop.
\item [\textit{y}]\ \\
  Vertical position of the window, relative to the upper left corner of
  the desktop.
\end{description}
\end{refparameters}

\begin{refreturn}
none
\end{refreturn}

\begin{refdescription}
The function changes the position of an opened window. It does not have
any effect on a fullscreen window.
\end{refdescription}


%-------------------------------------------------------------------------
\subsection{glfwGetWindowSize}

\textbf{C language syntax}
\begin{lstlisting}
void glfwGetWindowSize( int *width, int *height )
\end{lstlisting}

\begin{refparameters}
\begin{description}
\item [\textit{width}]\ \\
  Pointer to an integer that will hold the width of the window.
\item [\textit{height}]\ \\
  Pointer to an integer that will hold the height of the window.
\end{description}
\end{refparameters}

\begin{refreturn}
The current width and height of the opened window is returned in the
\textit{width} and \textit{height} parameters, respectively.
\end{refreturn}

\begin{refdescription}
The function is used for determining the size of an opened window.
The returned values are dimensions of the client area of the window
(i.e. excluding any window borders and decorations).
\end{refdescription}

\begin{refnotes}
Even if the size of a fullscreen window does not change once the window
has been opened, it does not necessarily have to be the same as the size
that was requested using \textbf{glfwOpenWindow}. Therefor it is wise to
use this function to determine the true size of the window once it has
been opened.
\end{refnotes}


%-------------------------------------------------------------------------
\subsection{glfwSetWindowSizeCallback}

\textbf{C language syntax}
\begin{lstlisting}
void glfwSetWindowSizeCallback( GLFWwindowsizefun cbfun )
\end{lstlisting}

\begin{refparameters}
\begin{description}
\item [\textit{cbfun}]\ \\
  Pointer to a callback function that will be called every time the
  window size changes. The function should have the following C language
  prototype:

  \texttt{void GLFWCALL functionname( int width, int height );}

  Where \textit{functionname} is the name of the callback function, and
  \textit{width} and \textit{height} are the dimensions of the window
  client area.

  If \textit{cbfun} is NULL, any previously selected callback function
  will be deselected.
\end{description}
\end{refparameters}

\begin{refreturn}
none
\end{refreturn}

\begin{refdescription}
The function selects which function to be called upon a window size
change event.

A window has to be opened for this function to have any effect.
\end{refdescription}

\begin{refnotes}
Window size changes are recorded continuously, but only reported when
\textbf{glfwPollEvents} or \textbf{glfwSwapBuffers} is called.

\end{refnotes}


%-------------------------------------------------------------------------
\subsection{glfwIconifyWindow}

\textbf{C language syntax}
\begin{lstlisting}
void glfwIconifyWindow( void )
\end{lstlisting}

\begin{refparameters}
none
\end{refparameters}

\begin{refreturn}
none
\end{refreturn}

\begin{refdescription}
Iconify a window. If the window is in fullscreen mode, then the desktop
video mode will be restored.
\end{refdescription}


%-------------------------------------------------------------------------
\subsection{glfwRestoreWindow}

\textbf{C language syntax}
\begin{lstlisting}
void glfwRestoreWindow( void )
\end{lstlisting}

\begin{refparameters}
none
\end{refparameters}

\begin{refreturn}
none
\end{refreturn}

\begin{refdescription}
Restore an iconified window. If the window that is restored is in
fullscreen mode, then the fullscreen video mode will be restored.
\end{refdescription}


%-------------------------------------------------------------------------
\begin{table}[p]
\begin{center}
\begin{tabular}{|l|p{9.5cm}|} \hline \raggedright
\textbf{Name}            & \textbf{Description} \\ \hline
GLFW\_OPENED             & GL\_TRUE if window is opened, else GL\_FALSE.\\ \hline
GLFW\_ACTIVE             & GL\_TRUE if window has focus, else GL\_FALSE.\\ \hline
GLFW\_ICONIFIED          & GL\_TRUE if window is iconified, else GL\_FALSE.\\ \hline
GLFW\_ACCELERATED        & GL\_TRUE if window is hardware accelerated, else GL\_FALSE.\\ \hline
GLFW\_RED\_BITS          & Number of bits for the red color component.\\ \hline
GLFW\_GREEN\_BITS        & Number of bits for the green color component.\\ \hline
GLFW\_BLUE\_BITS         & Number of bits for the blue color component.\\ \hline
GLFW\_ALPHA\_BITS        & Number of bits for the alpha buffer.\\ \hline
GLFW\_DEPTH\_BITS        & Number of bits for the depth buffer.\\ \hline
GLFW\_STENCIL\_BITS      & Number of bits for the stencil buffer.\\ \hline
GLFW\_REFRESH\_RATE      & Vertical monitor refresh rate in Hz. Zero indicates an unknown or a default refresh rate.\\ \hline
GLFW\_ACCUM\_RED\_BITS   & Number of bits for the red channel of the accumulator buffer.\\ \hline
GLFW\_ACCUM\_GREEN\_BITS & Number of bits for the green channel of the accumulator buffer.\\ \hline
GLFW\_ACCUM\_BLUE\_BITS  & Number of bits for the blue channel of the accumulator buffer.\\ \hline
GLFW\_ACCUM\_ALPHA\_BITS & Number of bits for the alpha channel of the accumulator buffer.\\ \hline
GLFW\_AUX\_BUFFERS       & Number of auxiliary buffers.\\ \hline
GLFW\_STEREO             & GL\_TRUE if stereo rendering is supported, else GL\_FALSE.\\ \hline
\end{tabular}
\end{center}
\caption{Window parameters for \textbf{glfwGetWindowParam}}
\label{tab:winparams}
\end{table}


%-------------------------------------------------------------------------
\subsection{glfwGetWindowParam}

\textbf{C language syntax}
\begin{lstlisting}
int glfwGetWindowParam( int param )
\end{lstlisting}

\begin{refparameters}
\begin{description}
\item [\textit{param}]\ \\
  A token selecting which parameter the function should return (see
  table \ref{tab:winparams}).
\end{description}
\end{refparameters}

\begin{refreturn}
The function returns different parameters depending on the value of
\textit{param}. Table \ref{tab:winparams} lists valid \textit{param}
values, and their corresponding return values.
\end{refreturn}

\begin{refdescription}
The function is used for acquiring various properties of an opened window.
\end{refdescription}

\begin{refnotes}
GLFW\_ACCELERATED is only supported under Windows. Other systems will
always return GL\_TRUE. Under Windows, GLFW\_ACCELERATED means that the
\OpenGL\ renderer is a 3rd party renderer, rather than the fallback
Microsoft software \OpenGL\ renderer. In other words, it is not a real
guarantee that the \OpenGL\ renderer is actually hardware accelerated.

GLFW\_REFRESH\_RATE is only supported under Windows, XFree86 and AmigaOS.
Other systems will always return zero (0). With some Windows drivers, zero
(0) may be returned, indicating a default refresh rate.
\end{refnotes}


%-------------------------------------------------------------------------
\subsection{glfwSwapBuffers}

\textbf{C language syntax}
\begin{lstlisting}
void glfwSwapBuffers( void )
\end{lstlisting}

\begin{refparameters}
none
\end{refparameters}

\begin{refreturn}
none
\end{refreturn}

\begin{refdescription}
The function swaps the back and front color buffers of the window. If
GLFW\_AUTO\_POLL\_EVENTS is enabled (which is the default),
\textbf{glfwPollEvents} is called before swapping the front and back
buffers.
\end{refdescription}


%-------------------------------------------------------------------------
\subsection{glfwSwapInterval}

\textbf{C language syntax}
\begin{lstlisting}
void glfwSwapInterval( int interval )
\end{lstlisting}

\begin{refparameters}
\begin{description}
\item [\textit{interval}]\ \\
  Minimum number of monitor vertical retraces between each buffer swap
  performed by \textbf{glfwSwapBuffers}. If \textit{interval} is zero,
  buffer swaps will not be synchronized to the vertical refresh of the
  monitor (also known as 'VSync off').
\end{description}
\end{refparameters}

\begin{refreturn}
none
\end{refreturn}

\begin{refdescription}
The function selects the minimum number of monitor vertical retraces that
should occur between two buffer swaps. If the selected swap interval is
one, the rate of buffer swaps will never be higher than the vertical
refresh rate of the monitor. If the selected swap interval is zero, the
rate of buffer swaps is only limited by the speed of the software and
the hardware.
\end{refdescription}

\begin{refnotes}
This function will only have an effect on hardware and drivers that
support user selection of the swap interval.
\end{refnotes}


%-------------------------------------------------------------------------
\pagebreak
\section{Video Modes}
Since \GLFW\ supports video mode changes when using a fullscreen window,
it also provides functionality for querying which video modes are
supported on a system.


%-------------------------------------------------------------------------
\subsection{glfwGetVideoModes}

\textbf{C language syntax}
\begin{lstlisting}
int glfwGetVideoModes( GLFWvidmode *list, int maxcount )
\end{lstlisting}

\begin{refparameters}
\begin{description}
\item [\textit{list}]\ \\
  A vector of \textit{GLFWvidmode} structures, which will be filled out
  by the function.
\item [\textit{maxcount}]\ \\
  Maximum number of video modes that \textit{list} vector can hold.
\end{description}
\end{refparameters}

\begin{refreturn}
The function returns the number of detected video modes (this number
will never exceed \textit{maxcount}). The \textit{list} vector is
filled out with the video modes that are supported by the system.
\end{refreturn}

\begin{refdescription}
The function returns a list of supported video modes. Each video mode is
represented by a \textit{GLFWvidmode} structure, which has the following
definition:

\begin{lstlisting}
typedef struct {
    int Width, Height; // Video resolution
    int RedBits;       // Number of red bits
    int GreenBits;     // Number of green bits
    int BlueBits;      // Number of blue bits
} GLFWvidmode;
\end{lstlisting}
\end{refdescription}

\begin{refnotes}
The returned list is sorted, first by color depth ($RedBits + GreenBits +
BlueBits$), and then by resolution ($Width \times Height$), with the
lowest resolution, fewest bits per pixel mode first.
\end{refnotes}


%-------------------------------------------------------------------------
\subsection{glfwGetDesktopMode}

\textbf{C language syntax}
\begin{lstlisting}
void glfwGetDesktopMode( GLFWvidmode *mode )
\end{lstlisting}

\begin{refparameters}
\begin{description}
\item [\textit{mode}]\ \\
  Pointer to a \textit{GLFWvidmode} structure, which will be filled out
  by the function.
\end{description}
\end{refparameters}

\begin{refreturn}
The \textit{GLFWvidmode} structure pointed to by \textit{mode} is filled
out with the desktop video mode.
\end{refreturn}

\begin{refdescription}
The function returns the desktop video mode in a \textit{GLFWvidmode}
structure. See \textbf{glfwGetVideoModes} for a definition of the
\textit{GLFWvidmode} structure.
\end{refdescription}

\begin{refnotes}
The color depth of the desktop display is always reported as the number
of bits for each individual color component (red, green and blue), even
if the desktop is not using an RGB or RGBA color format. For instance, an
indexed 256 color display may report \textit{RedBits} = 3,
\textit{GreenBits} = 3 and \textit{BlueBits} = 2, which adds up to 8 bits
in total.

The desktop video mode is the video mode used by the desktop, \textit{not}
the current video mode (which may differ from the desktop video mode if
the \GLFW\ window is a fullscreen window).
\end{refnotes}


%-------------------------------------------------------------------------
\pagebreak
\section{Input Handling}
\GLFW\ supports three channels of user input: keyboard input, mouse input
and joystick input.

Keyboard and mouse input can be treated either as events, using callback
functions, or as state, using functions for polling specific keyboard and
mouse states. Regardless of which method is used, all keyboard and mouse
input is collected using window event polling.

Joystick input is asynchronous to the keyboard and mouse input, and does
not require event polling for keeping up to date joystick information.
Also, joystick input is independent of any window, so a window does not
have to be opened for joystick input to be used.


%-------------------------------------------------------------------------
\subsection{glfwPollEvents}

\textbf{C language syntax}
\begin{lstlisting}
void glfwPollEvents( void )
\end{lstlisting}

\begin{refparameters}
none
\end{refparameters}

\begin{refreturn}
none
\end{refreturn}

\begin{refdescription}
The function is used for polling for events, such as user input and
window resize events. Upon calling this function, all window state and
keyboard and mouse input state is updated. If any related callback
functions are registered, these are called during the call to
\textbf{glfwPollEvents}.
\end{refdescription}

\begin{refnotes}
\textbf{glfwPollEvents} is called implicitly from \textbf{glfwSwapBuffers}
if GLFW\_AUTO\_POLL\_EVENTS is enabled (default). Thus, if
\textbf{glfwSwapBuffers} is called frequently, which is normally the case,
there is no need to call \textbf{glfwPollEvents}.
\end{refnotes}


%-------------------------------------------------------------------------
\begin{table}[p]
\begin{center}
\begin{tabular}{|l|l|} \hline \raggedright
\textbf{Name}             & \textbf{Description} \\ \hline
GLFW\_KEY\_SPACE          & Space\\ \hline
GLFW\_KEY\_ESC            & Escape\\ \hline
GLFW\_KEY\_F\textit{n}    & Function key \textit{n} (\textit{n} can be in the range 1..25).\\ \hline
GLFW\_KEY\_UP             & Cursor up\\ \hline
GLFW\_KEY\_DOWN           & Cursor down\\ \hline
GLFW\_KEY\_LEFT           & Cursor left\\ \hline
GLFW\_KEY\_RIGHT          & Cursor right\\ \hline
GLFW\_KEY\_LSHIFT         & Left shift key\\ \hline
GLFW\_KEY\_RSHIFT         & Right shift key\\ \hline
GLFW\_KEY\_LCTRL          & Left control key\\ \hline
GLFW\_KEY\_RCTRL          & Right control key\\ \hline
GLFW\_KEY\_LALT           & Left alternate function key\\ \hline
GLFW\_KEY\_RALT           & Right alternate function key\\ \hline
GLFW\_KEY\_TAB            & Tabulator\\ \hline
GLFW\_KEY\_ENTER          & Enter\\ \hline
GLFW\_KEY\_BACKSPACE      & Backspace\\ \hline
GLFW\_KEY\_INSERT         & Insert\\ \hline
GLFW\_KEY\_DEL            & Delete\\ \hline
GLFW\_KEY\_PAGEUP         & Page up\\ \hline
GLFW\_KEY\_PAGEDOWN       & Page down\\ \hline
GLFW\_KEY\_HOME           & Home\\ \hline
GLFW\_KEY\_END            & End\\ \hline
GLFW\_KEY\_KP\_\textit{n} & Keypad numeric key \textit{n} (\textit{n} can be in the range 0..9).\\ \hline
GLFW\_KEY\_KP\_DIVIDE     & Keypad divide ($\div$)\\ \hline
GLFW\_KEY\_KP\_MULTIPLY   & Keypad multiply ($\times$)\\ \hline
GLFW\_KEY\_KP\_SUBTRACT   & Keypad subtract ($-$)\\ \hline
GLFW\_KEY\_KP\_ADD        & Keypad add ($+$)\\ \hline
GLFW\_KEY\_KP\_DECIMAL    & Keypad decimal (. or ,)\\ \hline
GLFW\_KEY\_KP\_EQUAL      & Keypad equal (=)\\ \hline
GLFW\_KEY\_KP\_ENTER      & Keypad enter\\ \hline
\end{tabular}
\end{center}
\caption{Special key identifiers}
\label{tab:keys}
\end{table}


%-------------------------------------------------------------------------
\subsection{glfwGetKey}

\textbf{C language syntax}
\begin{lstlisting}
int glfwGetKey( int key )
\end{lstlisting}

\begin{refparameters}
\begin{description}
\item [\textit{key}]\ \\
  A keyboard key identifier, which can be either an uppercase printable
  ISO~8859-1 (Latin~1) character (e.g. 'A', '3' or '.'), or a special key
  identifier. Table \ref{tab:keys} lists valid special key identifiers.
\end{description}
\end{refparameters}

\begin{refreturn}
The function returns GLFW\_PRESS if the key is held down, or GLFW\_RELEASE
if the key is not held down.
\end{refreturn}

\begin{refdescription}
The function queries the current state of a specific keyboard key. The
physical location of each key depends on the system keyboard layout
setting.
\end{refdescription}

\begin{refnotes}
The constant GLFW\_KEY\_SPACE is equal to 32, which is the ISO~8859-1 code
for space.

Not all key codes are supported on all systems. Also, while some keys are
available on some keyboard layouts, they may not be available on other
keyboard layouts.

For systems that do not distinguish between left and right versions of
modifier keys (shift, alt and control), the left version is used (e.g.
GLFW\_KEY\_LSHIFT).

A window must be opened for the function to have any effect, and
\textbf{glfwPollEvents} or \textbf{glfwSwapBuffers} must be called before
any keyboard events are recorded and reported by \textbf{glfwGetKey}.
\end{refnotes}


%-------------------------------------------------------------------------
\subsection{glfwGetMouseButton}

\textbf{C language syntax}
\begin{lstlisting}
int glfwGetMouseButton( int button )
\end{lstlisting}

\begin{refparameters}
\begin{description}
\item [\textit{button}]\ \\
  A mouse button identifier, which can be one of GLFW\_MOUSE\_BUTTON\_LEFT,
  GLFW\_MOUSE\_BUTTON\_RIGHT or  GLFW\_MOUSE\_BUTTON\_MIDDLE.
\end{description}
\end{refparameters}

\begin{refreturn}
The function returns GLFW\_PRESS if the mouse button is held down, or
GLFW\_RELEASE if the mouse button is not held down.
\end{refreturn}

\begin{refdescription}
The function queries the current state of a specific mouse button.
\end{refdescription}

\begin{refnotes}
A window must be opened for the function to have any effect, and
\textbf{glfwPollEvents} or \textbf{glfwSwapBuffers} must be called before
any mouse button events are recorded and reported by
\textbf{glfwGetMouseButton}.
\end{refnotes}


%-------------------------------------------------------------------------
\subsection{glfwGetMousePos}

\textbf{C language syntax}
\begin{lstlisting}
void glfwGetMousePos( int *xpos, int *ypos )
\end{lstlisting}

\begin{refparameters}
\begin{description}
\item [\textit{xpos}]\ \\
  Pointer to an integer that will be filled out with the horizontal
  position of the mouse.
\item [\textit{ypos}]\ \\
  Pointer to an integer that will be filled out with the vertical
  position of the mouse.
\end{description}
\end{refparameters}

\begin{refreturn}
The function returns the current mouse position in \textit{xpos} and
\textit{ypos}.
\end{refreturn}

\begin{refdescription}
The function returns the current mouse position. If the cursor is not
hidden, the mouse position is the cursor position, relative to the upper
left corner of the window and limited to the client area of the window.
If the cursor is hidden, the mouse position is a virtual absolute
position, not limited to any boundaries except to those implied by the
maximum number that can be represented by a signed integer (normally
-2147483648 to +2147483647).
\end{refdescription}

\begin{refnotes}
A window must be opened for the function to have any effect, and
\textbf{glfwPollEvents} or \textbf{glfwSwapBuffers} must be called before
any mouse movements are recorded and reported by \textbf{glfwGetMousePos}.
\end{refnotes}


%-------------------------------------------------------------------------
\subsection{glfwSetMousePos}

\textbf{C language syntax}
\begin{lstlisting}
void glfwSetMousePos( int xpos, int ypos )
\end{lstlisting}

\begin{refparameters}
\begin{description}
\item [\textit{xpos}]\ \\
  Horizontal position of the mouse.
\item [\textit{ypos}]\ \\
  Vertical position of the mouse.
\end{description}
\end{refparameters}

\begin{refreturn}
none
\end{refreturn}

\begin{refdescription}
The function changes the position of the mouse. If the cursor is
visible (not disabled), the cursor will be moved to the specified
position, relative to the upper left corner of the window client area.
If the cursor is hidden (disabled), only the mouse position that is
reported by \GLFW\ is changed.
\end{refdescription}


%-------------------------------------------------------------------------
\subsection{glfwGetMouseWheel}

\textbf{C language syntax}
\begin{lstlisting}
int glfwGetMouseWheel( void )
\end{lstlisting}

\begin{refparameters}
none
\end{refparameters}

\begin{refreturn}
The function returns the current mouse wheel position.
\end{refreturn}

\begin{refdescription}
The function returns the current mouse wheel position. The mouse wheel can
be thought of as a third mouse axis, which is available as a separate
wheel or up/down stick on some mice.
\end{refdescription}

\begin{refnotes}
A window must be opened for the function to have any effect, and
\textbf{glfwPollEvents} or \textbf{glfwSwapBuffers} must be called before
any mouse wheel movements are recorded and reported by
\textbf{glfwGetMouseWheel}.
\end{refnotes}


%-------------------------------------------------------------------------
\subsection{glfwSetMouseWheel}

\textbf{C language syntax}
\begin{lstlisting}
void glfwSetMousePos( int pos )
\end{lstlisting}

\begin{refparameters}
\begin{description}
\item [\textit{pos}]\ \\
  Position of the mouse wheel.
\end{description}
\end{refparameters}

\begin{refreturn}
none
\end{refreturn}

\begin{refdescription}
The function changes the position of the mouse wheel.
\end{refdescription}


%-------------------------------------------------------------------------
\subsection{glfwSetKeyCallback}

\textbf{C language syntax}
\begin{lstlisting}
void glfwSetKeyCallback( GLFWkeyfun cbfun )
\end{lstlisting}

\begin{refparameters}
\begin{description}
\item [\textit{cbfun}]\ \\
  Pointer to a callback function that will be called every time a key is
  pressed or released. The function should have the following C language
  prototype:

  \texttt{void GLFWCALL functionname( int key, int action );}

  Where \textit{functionname} is the name of the callback function,
  \textit{key} is a key identifier, which is an uppercase printable
  ISO~8859-1 character or a special key identifier (see  table
  \ref{tab:keys}), and \textit{action} is either GLFW\_PRESS or
  GLFW\_RELEASE.

  If \textit{cbfun} is NULL, any previously selected callback function
  will be deselected.
\end{description}
\end{refparameters}

\begin{refreturn}
none
\end{refreturn}

\begin{refdescription}
The function selects which function to be called upon a keyboard key
event. The callback function is called every time the state of a single
key is changed (from released to pressed or vice versa). The reported keys
are unaffected by any modifiers (such as shift or alt).

A window has to be opened for this function to have any effect.
\end{refdescription}

\begin{refnotes}
Keyboard events are recorded continuously, but only reported when
\textbf{glfwPollEvents} or \textbf{glfwSwapBuffers} is called.
\end{refnotes}


%-------------------------------------------------------------------------
\subsection{glfwSetCharCallback}

\textbf{C language syntax}
\begin{lstlisting}
void glfwSetCharCallback( GLFWcharfun cbfun )
\end{lstlisting}

\begin{refparameters}
\begin{description}
\item [\textit{cbfun}]\ \\
  Pointer to a callback function that will be called every time a
  printable character is generated by the keyboard. The function should
  have the following C language prototype:

  \texttt{void GLFWCALL functionname( int character, int action );}

  Where \textit{functionname} is the name of the callback function,
  \textit{character} is a Unicode (ISO~10646) character, and
  \textit{action} is either GLFW\_PRESS or GLFW\_RELEASE.

  If \textit{cbfun} is NULL, any previously selected callback function
  will be deselected.
\end{description}
\end{refparameters}

\begin{refreturn}
none
\end{refreturn}

\begin{refdescription}
The function selects which function to be called upon a keyboard character
event. The callback function is called every time a key that results in a
printable Unicode character is pressed or released. Characters are
affected by modifiers (such as shift or alt).

A window has to be opened for this function to have any effect.
\end{refdescription}

\begin{refnotes}
Character events are recorded continuously, but only reported when
\textbf{glfwPollEvents} or \textbf{glfwSwapBuffers} is called.

Control characters, such as tab and carriage return, are not reported to
the character callback function, since they are not part of the Unicode
character set. Use the key callback function for such events (see
\textbf{glfwSetKeyCallback}).

The Unicode character set supports character codes above 255, so never
cast a Unicode character to an eight bit data type (e.g. the C language
'char' type) without first checking that the character code is less than
256. Also note that Unicode character codes 0 to 255 are equal to
ISO~8859-1 (Latin~1).
\end{refnotes}


%-------------------------------------------------------------------------
\subsection{glfwSetMouseButtonCallback}

\textbf{C language syntax}
\begin{lstlisting}
void glfwSetMouseButtonCallback( GLFWmousebuttonfun cbfun )
\end{lstlisting}

\begin{refparameters}
\begin{description}
\item [\textit{cbfun}]\ \\
  Pointer to a callback function that will be called every time a mouse
  button is pressed or released. The function should have the following C
  language prototype:

  \texttt{void GLFWCALL functionname( int button, int action );}

  Where \textit{functionname} is the name of the callback function,
  \textit{button} is a mouse button identifier (GLFW\_MOUSE\_BUTTON\_LEFT,
  GLFW\_MOUSE\_BUTTON\_RIGHT, or GLFW\_MOUSE\_BUTTON\_MIDDLE), and
  \textit{action} is either GLFW\_PRESS or GLFW\_RELEASE.

  If \textit{cbfun} is NULL, any previously selected callback function
  will be deselected.
\end{description}
\end{refparameters}

\begin{refreturn}
none
\end{refreturn}

\begin{refdescription}
The function selects which function to be called upon a mouse button
event.

A window has to be opened for this function to have any effect.
\end{refdescription}

\begin{refnotes}
Mouse button events are recorded continuously, but only reported when
\textbf{glfwPollEvents} or \textbf{glfwSwapBuffers} is called.
\end{refnotes}


%-------------------------------------------------------------------------
\subsection{glfwSetMousePosCallback}

\textbf{C language syntax}
\begin{lstlisting}
void glfwSetMousePosCallback( GLFWmouseposfun cbfun )
\end{lstlisting}

\begin{refparameters}
\begin{description}
\item [\textit{cbfun}]\ \\
  Pointer to a callback function that will be called every time the mouse
  is moved. The function should have the following C language prototype:

  \texttt{void GLFWCALL functionname( int x, int y );}

  Where \textit{functionname} is the name of the callback function, and
  \textit{x} and \textit{y} are the mouse coordinates (see
  \textbf{glfwGetMousePos} for more information on mouse coordinates).

  If \textit{cbfun} is NULL, any previously selected callback function
  will be deselected.
\end{description}
\end{refparameters}

\begin{refreturn}
none
\end{refreturn}

\begin{refdescription}
The function selects which function to be called upon a mouse motion event.

A window has to be opened for this function to have any effect.
\end{refdescription}

\begin{refnotes}
Mouse motion events are recorded continuously, but only reported when
\textbf{glfwPollEvents} or \textbf{glfwSwapBuffers} is called.
\end{refnotes}


%-------------------------------------------------------------------------
\subsection{glfwSetMouseWheelCallback}

\textbf{C language syntax}
\begin{lstlisting}
void glfwSetMouseWheelCallback( GLFWmousewheelfun cbfun )
\end{lstlisting}

\begin{refparameters}
\begin{description}
\item [\textit{cbfun}]\ \\
  Pointer to a callback function that will be called every time the mouse
  wheel is moved. The function should have the following C language
  prototype:

  \texttt{void GLFWCALL functionname( int pos );}

  Where \textit{functionname} is the name of the callback function, and
  \textit{pos} is the mouse wheel position.

  If \textit{cbfun} is NULL, any previously selected callback function
  will be deselected.
\end{description}
\end{refparameters}

\begin{refreturn}
none
\end{refreturn}

\begin{refdescription}
The function selects which function to be called upon a mouse wheel event.

A window has to be opened for this function to have any effect.
\end{refdescription}

\begin{refnotes}
Mouse wheel events are recorded continuously, but only reported when
\textbf{glfwPollEvents} or \textbf{glfwSwapBuffers} is called.
\end{refnotes}


%-------------------------------------------------------------------------
\begin{table}[p]
\begin{center}
\begin{tabular}{|l|l|}\hline \raggedright
\textbf{Name} & \textbf{Return value}\\ \hline
GLFW\_PRESENT & GL\_TRUE if the joystick is connected, else GL\_FALSE.\\ \hline
GLFW\_AXES    & Number of axes supported by the joystick.\\ \hline
GLFW\_BUTTONS & Number of buttons supported by the joystick.\\ \hline
\end{tabular}
\end{center}
\caption{Joystick parameters for \textbf{glfwGetJoystickParam}}
\label{tab:joyparams}
\end{table}


%-------------------------------------------------------------------------
\subsection{glfwGetJoystickParam}

\textbf{C language syntax}
\begin{lstlisting}
int glfwGetJoystickParam( int joy, int param )
\end{lstlisting}

\begin{refparameters}
\begin{description}
\item [\textit{joy}]\ \\
  A joystick identifier, which should be GLFW\_JOYSTICK\_\textit{n}, where
  \textit{n} is in the range 1 to 16.
\item [\textit{param}]\ \\
  A token selecting which parameter the function should return (see table
  \ref{tab:joyparams}).
\end{description}
\end{refparameters}

\begin{refreturn}
The function returns different parameters depending on the value of
\textit{param}. Table \ref{tab:joyparams} lists valid \textit{param}
values, and their corresponding return values.
\end{refreturn}

\begin{refdescription}
The function is used for acquiring various properties of a joystick.
\end{refdescription}

\begin{refnotes}
The joystick information is updated every time the function is called.

No window has to be opened for joystick information to be valid.
\end{refnotes}


%-------------------------------------------------------------------------
\subsection{glfwGetJoystickPos}

\textbf{C language syntax}
\begin{lstlisting}
int glfwGetJoystickPos( int joy, float *pos, int numaxes )
\end{lstlisting}

\begin{refparameters}
\begin{description}
\item [\textit{joy}]\ \\
  A joystick identifier, which should be GLFW\_JOYSTICK\_\textit{n}, where
  \textit{n} is in the range 1 to 16.
\item [\textit{pos}]\ \\
  An array that will hold the positional values for all requested axes.
\item [\textit{numaxes}]\ \\
  Specifies how many axes should be returned.
\end{description}
\end{refparameters}

\begin{refreturn}
The function returns the number of actually returned axes. This is the
minimum of \textit{numaxes} and the number of axes supported by the
joystick. If the joystick is not supported or connected, the function will
return 0 (zero).
\end{refreturn}

\begin{refdescription}
The function queries the current position of one or more axes of a
joystick. The positional values are returned in an array, where the first
element represents the first axis of the joystick (normally the X axis).
Each position is in the range -1.0 to 1.0. Where applicable, the positive
direction of an axis is right, forward or up, and the negative direction
is left, back or down.

If \textit{numaxes} exceeds the number of axes supported by the joystick,
or if the joystick is not available, the unused elements in the
\textit{pos} array will be set to 0.0 (zero).
\end{refdescription}

\begin{refnotes}
The joystick state is updated every time the function is called, so there
is no need to call \textbf{glfwPollEvents} for joystick state to be
updated.

Use \textbf{glfwGetJoystickParam} to retrieve joystick capabilities, such
as joystick availability and number of supported axes.

No window has to be opened for joystick input to be valid.
\end{refnotes}


%-------------------------------------------------------------------------
\subsection{glfwGetJoystickButtons}

\textbf{C language syntax}
\begin{lstlisting}
int glfwGetJoystickButtons( int joy, unsigned char *buttons,
                            int numbuttons )
\end{lstlisting}

\begin{refparameters}
\begin{description}
\item [\textit{joy}]\ \\
  A joystick identifier, which should be GLFW\_JOYSTICK\_\textit{n}, where
  \textit{n} is in the range 1 to 16.
\item [\textit{buttons}]\ \\
  An array that will hold the button states for all requested buttons.
\item [\textit{numbuttons}]\ \\
  Specifies how many buttons should be returned.
\end{description}
\end{refparameters}

\begin{refreturn}
The function returns the number of actually returned buttons. This is the
minimum of \textit{numbuttons} and the number of buttons supported by the
joystick. If the joystick is not supported or connected, the function will
return 0 (zero).
\end{refreturn}

\begin{refdescription}
The function queries the current state of one or more buttons of a
joystick. The button states are returned in an array, where the first
element represents the first button of the joystick. Each state can be
either GLFW\_PRESS or GLFW\_RELEASE.

If \textit{numbuttons} exceeds the number of buttons supported by the
joystick, or if the joystick is not available, the unused elements in the
\textit{buttons} array will be set to GLFW\_RELEASE.
\end{refdescription}

\begin{refnotes}
The joystick state is updated every time the function is called, so there
is no need to call \textbf{glfwPollEvents} for joystick state to be
updated.

Use \textbf{glfwGetJoystickParam} to retrieve joystick capabilities, such
as joystick availability and number of supported buttons.

No window has to be opened for joystick input to be valid.
\end{refnotes}


%-------------------------------------------------------------------------
\pagebreak
\section{Timing}

%-------------------------------------------------------------------------
\subsection{glfwGetTime}

\textbf{C language syntax}
\begin{lstlisting}
double glfwGetTime( void )
\end{lstlisting}

\begin{refparameters}
none
\end{refparameters}

\begin{refreturn}
The function returns the value of the high precision timer. The time is
measured in seconds, and is returned as a double precision floating point
value.
\end{refreturn}

\begin{refdescription}
The function returns the state of a high precision timer. Unless the timer
has been set by the \textbf{glfwSetTime} function, the time is measured as
the number of seconds that have passed since \textbf{glfwInit} was called.
\end{refdescription}

\begin{refnotes}
The resolution of the timer depends on which system the program is running
on. The worst case resolution is somewhere in the order of $10~ms$, while
for most systems the resolution should be better than $1~\mu s$.
\end{refnotes}


%-------------------------------------------------------------------------
\subsection{glfwSetTime}

\textbf{C language syntax}
\begin{lstlisting}
void glfwSetTime( double time )
\end{lstlisting}

\begin{refparameters}
\begin{description}
\item [\textit{time}]\ \\
  Time (in seconds) that the timer should be set to.
\end{description}
\end{refparameters}

\begin{refreturn}
none
\end{refreturn}

\begin{refdescription}
The function sets the current time of the high precision timer to the
specified time. Subsequent calls to \textbf{glfwGetTime} will be relative
to this time. The time is given in seconds.
\end{refdescription}


%-------------------------------------------------------------------------
\subsection{glfwSleep}

\textbf{C language syntax}
\begin{lstlisting}
void glfwSleep( double time )
\end{lstlisting}

\begin{refparameters}
\begin{description}
\item [\textit{time}]\ \\
  Time, in seconds, to sleep.
\end{description}
\end{refparameters}

\begin{refreturn}
none
\end{refreturn}

\begin{refdescription}
The function puts the calling thread to sleep for the requested period of
time. Only the calling thread is put to sleep. Other threads within the
same process can still execute.
\end{refdescription}

\begin{refnotes}
There is usually a system dependent minimum time for which it is possible
to sleep. This time is generally in the range 1~$ms$ to 20~$ms$, depending
on thread sheduling time slot intervals etc. Using a shorter time as a
parameter to \textbf{glfwSleep} can give one of two results: either the
thread will sleep for the minimum possible sleep time, or the thread will
not sleep at all (\textbf{glfwSleep} returns immediately). The latter
should only happen when very short sleep times are specified, if at all.
\end{refnotes}


%-------------------------------------------------------------------------
\pagebreak
\section{Image and Texture Loading}
In order to aid texture file loading, \GLFW\ has basic support for loading
images from files.


%-------------------------------------------------------------------------
\begin{table}[p]
\begin{center}
\begin{tabular}{|l|p{9.0cm}|} \hline \raggedright
\textbf{Name}          & \textbf{Description}\\ \hline
GLFW\_NO\_RESCALE\_BIT & Do not rescale image to closest $2^m\times2^n$ resolution.\\ \hline
GLFW\_ORIGIN\_UL\_BIT  & Specifies that the origin of the \textit{loaded} image should be in the upper left corner (default is the lower left corner)\\ \hline
\end{tabular}
\end{center}
\caption{Flags for \textbf{glfwReadImage}}
\label{tab:rdimgflags}
\end{table}


%-------------------------------------------------------------------------
\subsection{glfwReadImage}

\textbf{C language syntax}
\begin{lstlisting}
int glfwReadImage( const char *name, GLFWimage *img, int flags )
\end{lstlisting}

\begin{refparameters}
\begin{description}
\item [\textit{name}]\ \\
  A null terminated ISO~8859-1 string holding the name of the file that
  should be read.
\item [\textit{img}]\ \\
  Pointer to a GLFWimage struct, which will hold the information about
  the loaded image (if the read was successful).
\item [\textit{flags}]\ \\
  Flags for controlling the image reading process. Valid flags are listed
  in table \ref{tab:rdimgflags}
\end{description}
\end{refparameters}

\begin{refreturn}
The function returns GL\_TRUE if the image was loaded successfully.
Otherwise GL\_FALSE is returned.
\end{refreturn}

\begin{refdescription}
The function reads an image from the file specified by the parameter
\textit{name} and returns the image information and data in a GLFWimage
structure, which has the following definition:

\begin{lstlisting}
typedef struct {
    int Width, Height;    // Image dimensions
    int Format;           // OpenGL pixel format
    int BytesPerPixel;    // Number of bytes per pixel
    unsigned char *Data;  // Pointer to pixel data
} GLFWimage;
\end{lstlisting}

\textit{Width} and \textit{Height} give the dimensions of the image.
\textit{Format} specifies an \OpenGL\ pixel format, which can be
GL\_LUMINANCE (for gray scale images), GL\_RGB or GL\_RGBA.
\textit{BytesPerPixel} specifies the number of bytes per pixel.
\textit{Data} is a pointer to the actual pixel data.

By default the read image is rescaled to the nearest larger $2^m\times2^n$
resolution using bilinear interpolation, if necessary, which is useful if
the image is to be used as an \OpenGL\ texture. This behavior can be
disabled by setting the GLFW\_NO\_RESCALE\_BIT flag.

Unless the flag GLFW\_ORIGIN\_UL\_BIT is set, the first pixel in
\textit{img->Data} is the lower left corner of the image. If the flag
GLFW\_ORIGIN\_UL\_BIT is set, however, the first pixel is the upper left
corner.
\end{refdescription}

\begin{refnotes}
\textbf{glfwReadImage} supports the Truevision Targa version 1 file format
(.TGA). Supported pixel formats are: 8-bit gray scale, 8-bit paletted
(24/32-bit color), 24-bit true color and 32-bit true color + alpha.

Paletted images are translated into true color or true color + alpha pixel
formats.
\end{refnotes}


%-------------------------------------------------------------------------
\subsection{glfwFreeImage}

\textbf{C language syntax}
\begin{lstlisting}
void glfwFreeImage( GLFWimage *img )
\end{lstlisting}

\begin{refparameters}
\begin{description}
\item [\textit{img}]\ \\
  Pointer to a GLFWimage struct.
\end{description}
\end{refparameters}

\begin{refreturn}
none
\end{refreturn}

\begin{refdescription}
The function frees any memory occupied by a loaded image, and clears all
the fields of the GLFWimage struct. Any image that has been loaded by the
\textbf{glfwReadImage} function should be deallocated using this function,
once the image is not needed anymore.
\end{refdescription}


%-------------------------------------------------------------------------
\begin{table}[p]
\begin{center}
\begin{tabular}{|l|p{9.0cm}|} \hline \raggedright
\textbf{Name}             & \textbf{Description}\\ \hline
GLFW\_BUILD\_MIPMAPS\_BIT & Automatically build and upload all mipmap levels.\\ \hline
GLFW\_ORIGIN\_UL\_BIT     & Specifies that the origin of the \textit{loaded} image should be in the upper left corner (default is the lower left corner)\\ \hline
\end{tabular}
\end{center}
\caption{Flags for \textbf{glfwLoadTexture2D}}
\label{tab:ldtexflags}
\end{table}


%-------------------------------------------------------------------------
\subsection{glfwLoadTexture2D}

\textbf{C language syntax}
\begin{lstlisting}
nt glfwLoadTexture2D( const char *name, int flags )
\end{lstlisting}

\begin{refparameters}
\begin{description}
\item [\textit{name}]\ \\
  An ISO~8859-1 string holding the name of the file that should be loaded.
\item [\textit{flags}]\ \\
  Flags for controlling the texture loading process. Valid flags are
  listed in table \ref{tab:ldtexflags}.
\end{description}
\end{refparameters}

\begin{refreturn}
The function returns GL\_TRUE if the texture was loaded successfully.
Otherwise GL\_FALSE is returned.
\end{refreturn}

\begin{refdescription}
The function reads an image from the file specified by the parameter
\textit{name} and uploads the image to \OpenGL\ texture memory (using the
\textbf{glTexImage2D} function).

If the GLFW\_BUILD\_MIPMAPS\_BIT flag is set, all mipmap levels for the
loaded texture are generated and uploaded to texture memory.

Unless the flag GLFW\_ORIGIN\_UL\_BIT is set, the first pixel in
\textit{img->Data} is the lower left corner of the image. If the flag
GLFW\_ORIGIN\_UL\_BIT is set, however, the first pixel is the upper left
corner.
\end{refdescription}

\begin{refnotes}
\textbf{glfwLoadTexture2D} supports the Truevision Targa version 1 file
format (.TGA). Supported pixel formats are: 8-bit gray scale, 8-bit
paletted (24/32-bit color), 24-bit true color and 32-bit true color +
alpha.

Paletted images are translated into true color or true color + alpha pixel
formats.

The read texture is always rescaled to the nearest larger $2^m\times2^n$
resolution using bilinear interpolation, if necessary, since \OpenGL\
requires textures to have a $2^m\times2^n$ resolution.

If the GL\_SGIS\_generate\_mipmap extension, which is usually hardware
accelerated, is supported by the \OpenGL\ implementation it will be used
for mipmap generation. Otherwise the mipmaps will be generated by \GLFW\
in software.
\end{refnotes}


%-------------------------------------------------------------------------
\pagebreak
\section{OpenGL Extension Support}
One of the great features of \OpenGL\ is its support for extensions, which
allow independent vendors to supply non-standard functionality in their
\OpenGL\ implementations. Using extensions is different under different
systems, which is why \GLFW\ has provided an operating system independent
interface to querying and using \OpenGL\ extensions.


%-------------------------------------------------------------------------
\subsection{glfwExtensionSupported}

\textbf{C language syntax}
\begin{lstlisting}
int glfwExtensionSupported( const char *extension )
\end{lstlisting}

\begin{refparameters}
\begin{description}
\item [\textit{extension}]\ \\
  A null terminated ISO~8859-1 string containing the name of an \OpenGL\
  extension.
\end{description}
\end{refparameters}

\begin{refreturn}
The function returns GL\_TRUE if the extension is supported. Otherwise it
returns GL\_FALSE.
\end{refreturn}

\begin{refdescription}
The function does a string search in the list of supported \OpenGL\
extensions to find if the specified extension is listed.
\end{refdescription}

\begin{refnotes}
An \OpenGL\ context must be created before this function can be called
(i.e. an \OpenGL\ window must have been opened with
\textbf{glfwOpenWindow}).

In addition to checking for \OpenGL\ extensions, \GLFW\ also checks for
extensions in the operating system ``glue API'', such as WGL extensions
under Windows and glX extensions under the X Window System.
\end{refnotes}


%-------------------------------------------------------------------------
\subsection{glfwGetProcAddress}

\textbf{C language syntax}
\begin{lstlisting}
void * glfwGetProcAddress( const char *procname )
\end{lstlisting}

\begin{refparameters}
\begin{description}
\item [\textit{procname}]\ \\
  A null terminated ISO~8859-1 string containing the name of an \OpenGL\
  extension function.
\end{description}
\end{refparameters}

\begin{refreturn}
The function returns the pointer to the specified \OpenGL\ function if it
is supported, otherwise NULL is returned.
\end{refreturn}

\begin{refdescription}
The function acquires the pointer to an \OpenGL\ extension function. Some
(but not all) \OpenGL\ extensions define new API functions, which are
usually not available through normal linking. It is therefore necessary to
get access to those API functions at runtime.
\end{refdescription}

\begin{refnotes}
An \OpenGL\ context must be created before this function can be called
(i.e. an \OpenGL\ window must have been opened with
\textbf{glfwOpenWindow}).

Some systems do not support dynamic function pointer retrieval, in which
case \textbf{glfwGetProcAddress} will always return NULL.
\end{refnotes}


%-------------------------------------------------------------------------
\subsection{glfwGetGLVersion}

\textbf{C language syntax}
\begin{lstlisting}
void glfwGetGLVersion( int *major, int *minor, int *rev )
\end{lstlisting}

\begin{refparameters}
\begin{description}
\item [\textit{major}]\ \\
  Pointer to an integer that will hold the major version number.
\item [\textit{minor}]\ \\
  Pointer to an integer that will hold the minor version number.
\item [\textit{rev}]\ \\
  Pointer to an integer that will hold the revision.
\end{description}
\end{refparameters}

\begin{refreturn}
The function returns the major and minor version numbers and the revision
for the currently used \OpenGL\ implementation.
\end{refreturn}

\begin{refdescription}
The function returns the \OpenGL\ implementation version. This is a
convenient function that parses the version number information from the
string returned by calling \texttt{glGetString(~GL\_VERSION~)}. The
\OpenGL\ version information can be used to determine what functionality
is supported by the used \OpenGL\ implementation.
\end{refdescription}

\begin{refnotes}
An \OpenGL\ context must be created before this function can be called
(i.e. an \OpenGL\ window must have been opened with
\textbf{glfwOpenWindow}).
\end{refnotes}


%-------------------------------------------------------------------------
\pagebreak
\section{Threads}
A thread is a separate execution path within a process. All threads within
a process share the same address space and resources. Threads execute in
parallel, either virtually by means of time-sharing on a single processor,
or truly in parallel on several processors. Even on a multi-processor
system, time-sharing is employed in order to maximize processor
utilization and to ensure fair scheduling. \GLFW\ provides an operating
system independent interface to thread management.


%-------------------------------------------------------------------------
\subsection{glfwCreateThread}

\textbf{C language syntax}
\begin{lstlisting}
GLFWthread glfwCreateThread( GLFWthreadfun fun, void *arg )
\end{lstlisting}

\begin{refparameters}
\begin{description}
\item [\textit{fun}]\ \\
  A pointer to a function that acts as the entry point for the new thread.
  The function should have the following C language prototype:

  \texttt{void GLFWCALL functionname( void *arg );}

  Where \textit{functionname} is the name of the thread function, and
  \textit{arg} is the user supplied argument (see below).
\item [\textit{arg}]\ \\
  An arbitrary argument for the thread. \textit{arg} will be passed as the
  argument to the thread function pointed to by \textit{fun}. For
  instance, \textit{arg} can point to data that is to be processed by the
  thread.
\end{description}
\end{refparameters}

\begin{refreturn}
The function returns a thread identification number if the thread was
created successfully. This number is always positive. If the function
fails, a negative number is returned.
\end{refreturn}

\begin{refdescription}
The function creates a new thread, which executes within the same address
space as the calling process. The thread entry point is specified with the
\textit{fun} argument.

Once the thread function \textit{fun} returns, the thread dies.
\end{refdescription}

\begin{refnotes}
Even if the function returns a positive thread ID, indicating that the
thread was created successfully, the thread may be unable to execute, for
instance if the thread start address is not a valid thread entry point.
\end{refnotes}


%-------------------------------------------------------------------------
\subsection{glfwDestroyThread}

\textbf{C language syntax}
\begin{lstlisting}
void glfwDestroyThread( GLFWthread ID )
\end{lstlisting}

\begin{refparameters}
\begin{description}
\item [\textit{ID}]\ \\
  A thread identification handle, which is returned by
  \textbf{glfwCreateThread} or \textbf{glfwGetThreadID}.
\end{description}
\end{refparameters}

\begin{refreturn}
none
\end{refreturn}

\begin{refdescription}
The function kills a running thread and removes it from the thread list.
\end{refdescription}

\begin{refnotes}
This function is a very dangerous operation, which may interrupt a thread
in the middle of an important operation, and its use is discouraged. You
should always try to end a thread in a graceful way using thread
communication, and use \textbf{glfwWaitThread} in order to wait for the
thread to die.
\end{refnotes}


%-------------------------------------------------------------------------
\subsection{glfwWaitThread}

\textbf{C language syntax}
\begin{lstlisting}
int glfwWaitThread( GLFWthread ID, int waitmode )
\end{lstlisting}

\begin{refparameters}
\begin{description}
\item [\textit{ID}]\ \\
  A thread identification handle, which is returned by
  \textbf{glfwCreateThread} or \textbf{glfwGetThreadID}.
\item [\textit{waitmode}]\ \\
  Can be either GLFW\_WAIT or GLFW\_NOWAIT.
\end{description}
\end{refparameters}

\begin{refreturn}
The function returns GL\_TRUE if the specified thread died after the
function was called, or the thread did not exist, in which case
\textbf{glfwWaitThread} will return immediately regardless of
\textit{waitmode}. The function returns GL\_FALSE if \textit{waitmode}
is GLFW\_NOWAIT, and the specified thread exists and is still running.
\end{refreturn}

\begin{refdescription}
If \textit{waitmode} is GLFW\_WAIT, the function waits for a thread to
die. If \textit{waitmode} is GLFW\_NOWAIT, the function checks if a thread
exists and returns immediately.
\end{refdescription}


%-------------------------------------------------------------------------
\subsection{glfwGetThreadID}

\textbf{C language syntax}
\begin{lstlisting}
GLFWthread glfwGetThreadID( void )
\end{lstlisting}

\begin{refparameters}
none
\end{refparameters}

\begin{refreturn}
The function returns a thread identification handle for the calling
thread.
\end{refreturn}

\begin{refdescription}
The function determines the thread ID for the calling thread. The ID is
the same value as was returned by \textbf{glfwCreateThread} when the
thread was created.
\end{refdescription}


%-------------------------------------------------------------------------
\pagebreak
\section{Mutexes}
Mutexes are used to securely share data between threads. A mutex object
can only be owned by one thread at a time. If more than one thread
requires access to a mutex object, all but one thread will be put to sleep
until they get access to it.


%-------------------------------------------------------------------------
\subsection{glfwCreateMutex}

\textbf{C language syntax}
\begin{lstlisting}
GLFWmutex glfwCreateMutex( void )
\end{lstlisting}

\begin{refparameters}
none
\end{refparameters}

\begin{refreturn}
The function returns a mutex handle, or NULL if the mutex could not be
created.
\end{refreturn}

\begin{refdescription}
The function creates a mutex object, which can be used to control access
to data that is shared between threads.
\end{refdescription}


%-------------------------------------------------------------------------
\subsection{glfwDestroyMutex}

\textbf{C language syntax}
\begin{lstlisting}
void glfwDestroyMutex( GLFWmutex mutex )
\end{lstlisting}

\begin{refparameters}
\begin{description}
\item [\textit{mutex}]\ \\
  A mutex object handle.
\end{description}
\end{refparameters}

\begin{refreturn}
none
\end{refreturn}

\begin{refdescription}
The function destroys a mutex object. After a mutex object has been
destroyed, it may no longer be used by any thread.
\end{refdescription}


%-------------------------------------------------------------------------
\subsection{glfwLockMutex}

\textbf{C language syntax}
\begin{lstlisting}
void glfwLockMutex( GLFWmutex mutex )
\end{lstlisting}

\begin{refparameters}
\begin{description}
\item [\textit{mutex}]\ \\
  A mutex object handle.
\end{description}
\end{refparameters}

\begin{refreturn}
none
\end{refreturn}

\begin{refdescription}
The function will acquire a lock on the selected mutex object. If the
mutex is already locked by another thread, the function will block the
calling thread until it is released by the locking thread. Once the
function returns, the calling thread has an exclusive lock on the mutex.
To release the mutex, call \textbf{glfwUnlockMutex}.
\end{refdescription}


%-------------------------------------------------------------------------
\subsection{glfwUnlockMutex}

\textbf{C language syntax}
\begin{lstlisting}
void glfwUnlockMutex( GLFWmutex mutex )
\end{lstlisting}

\begin{refparameters}
\begin{description}
\item [\textit{mutex}]\ \\
  A mutex object handle.
\end{description}
\end{refparameters}

\begin{refreturn}
none
\end{refreturn}

\begin{refdescription}
The function releases the lock of a locked mutex object.
\end{refdescription}


%-------------------------------------------------------------------------
\pagebreak
\section{Condition Variables}
Condition variables are used to synchronize threads. A thread can wait for
a condition variable to be signaled by another thread.


%-------------------------------------------------------------------------
\subsection{glfwCreateCond}

\textbf{C language syntax}
\begin{lstlisting}
GLFWcond glfwCreateCond( void )
\end{lstlisting}

\begin{refparameters}
none
\end{refparameters}

\begin{refreturn}
The function returns a condition variable handle, or NULL if the condition
variable could not be created.
\end{refreturn}

\begin{refdescription}
The function creates a condition variable object, which can be used to
synchronize threads.
\end{refdescription}


%-------------------------------------------------------------------------
\subsection{glfwDestroyCond}

\textbf{C language syntax}
\begin{lstlisting}
void glfwDestroyCond( GLFWcond cond )
\end{lstlisting}

\begin{refparameters}
\begin{description}
\item [\textit{cond}]\ \\
  A condition variable object handle.
\end{description}
\end{refparameters}

\begin{refreturn}
none
\end{refreturn}

\begin{refdescription}
The function destroys a condition variable object. After a condition
variable object has been destroyed, it may no longer be used by any
thread.
\end{refdescription}


%-------------------------------------------------------------------------
\subsection{glfwWaitCond}

\textbf{C language syntax}
\begin{lstlisting}
void glfwWaitCond( GLFWcond cond, GLFWmutex mutex, double timeout )
\end{lstlisting}

\begin{refparameters}
\begin{description}
\item [\textit{cond}]\ \\
  A condition variable object handle.
\item [\textit{mutex}]\ \\
  A mutex object handle.
\item [\textit{timeout}]\ \\
  Maximum time to wait for the condition variable. The parameter can
  either be a positive time (in seconds), or GLFW\_INFINITY.
\end{description}
\end{refparameters}

\begin{refreturn}
none
\end{refreturn}

\begin{refdescription}
The function atomically unlocks the mutex specified by \textit{mutex}, and
waits for the condition variable \textit{cond} to be signaled. The thread
execution is suspended and does not consume any CPU time until the
condition variable is signaled or the amount of time specified by timeout
has passed. If timeout is GLFW\_INFINITY, \textbf{glfwWaitCond} will wait
forever for \textit{cond} to be signaled. Before returning to the calling
thread, \textbf{glfwWaitCond} automatically re-acquires the mutex.
\end{refdescription}

\begin{refnotes}
The mutex specified by \textit{mutex} must be locked by the calling thread
before entrance to \textbf{glfwWaitCond}.

A condition variable must always be associated with a mutex, to avoid the
race condition where a thread prepares to wait on a condition variable and
another thread signals the condition just before the first thread actually
waits on it.
\end{refnotes}


%-------------------------------------------------------------------------
\subsection{glfwSignalCond}

\textbf{C language syntax}
\begin{lstlisting}
void glfwSignalCond( GLFWcond cond )
\end{lstlisting}

\begin{refparameters}
\begin{description}
\item [\textit{cond}]\ \\
  A condition variable object handle.
\end{description}
\end{refparameters}

\begin{refreturn}
none
\end{refreturn}

\begin{refdescription}
The function restarts one of the threads that are waiting on the condition
variable \textit{cond}. If no threads are waiting on \textit{cond},
nothing happens. If several threads are waiting on \textit{cond}, exactly
one is restarted, but it is not specified which.
\end{refdescription}

\begin{refnotes}
When several threads are waiting for the condition variable, which thread
is started depends on operating system scheduling rules, and may vary from
system to system and from time to time.
\end{refnotes}


%-------------------------------------------------------------------------
\subsection{glfwBroadcastCond}

\textbf{C language syntax}
\begin{lstlisting}
void glfwBroadcastCond( GLFWcond cond )
\end{lstlisting}

\begin{refparameters}
\begin{description}
\item [\textit{cond}]\ \\
  A condition variable object handle.
\end{description}
\end{refparameters}

\begin{refreturn}
none
\end{refreturn}

\begin{refdescription}
The function restarts all the threads that are waiting on the condition
variable \textit{cond}. If no threads are waiting on \textit{cond},
nothing happens.
\end{refdescription}

\begin{refnotes}
When several threads are waiting for the condition variable, the order in
which threads are started depends on operating system scheduling rules,
and may vary from system to system and from time to time.
\end{refnotes}


%-------------------------------------------------------------------------
\pagebreak
\section{Miscellaneous}


%-------------------------------------------------------------------------
\subsection{glfwEnable/glfwDisable}

\textbf{C language syntax}
\begin{lstlisting}
void glfwEnable( int token )
void glfwDisable( int token )
\end{lstlisting}

\begin{refparameters}
\begin{description}
\item [\textit{token}]\ \\
  A value specifying a feature to enable or disable. Valid tokens are
  listed in table \ref{tab:enable}.
\end{description}
\end{refparameters}

\begin{refreturn}
none
\end{refreturn}

\begin{refdescription}
\textbf{glfwEnable} is used to enable a certain feature, while
\textbf{glfwDisable} is used to disable it. Below follows a description of
each feature.
\end{refdescription}


\begin{table}[p]
\begin{center}
\begin{tabular}{|l|p{5.0cm}|p{3.0cm}|} \hline \raggedright
\textbf{Name} & \textbf{Controls} & \textbf{Default}\\ \hline
\hyperlink{lnk:autopollevents}{GLFW\_AUTO\_POLL\_EVENTS}         & Automatic event polling when \textbf{glfwSwapBuffers} is called & Enabled\\ \hline
\hyperlink{lnk:keyrepeat}{GLFW\_KEY\_REPEAT}                     & Keyboard key repeat                                    & Disabled\\ \hline
\hyperlink{lnk:mousecursor}{GLFW\_MOUSE\_CURSOR}                 & Mouse cursor visibility                                & Enabled in windowed mode. Disabled in fullscreen mode.\\ \hline
\hyperlink{lnk:stickykeys}{GLFW\_STICKY\_KEYS}                   & Keyboard key ``stickiness''                            & Disabled\\ \hline
\hyperlink{lnk:stickymousebuttons}{GLFW\_STICKY\_MOUSE\_BUTTONS} & Mouse button ``stickiness''                            & Disabled\\ \hline
\hyperlink{lnk:systemkeys}{GLFW\_SYSTEM\_KEYS}                   & Special system key actions                             & Enabled\\ \hline
\end{tabular}
\end{center}
\caption{Tokens for \textbf{glfwEnable}/\textbf{glfwDisable}}
\label{tab:enable}
\end{table}


\bigskip\begin{mysamepage}\hypertarget{lnk:autopollevents}{}
\textbf{GLFW\_AUTO\_POLL\_EVENTS}\\
When GLFW\_AUTO\_POLL\_EVENTS is enabled, \textbf{glfwPollEvents} is
automatically called each time that \textbf{glfwSwapBuffers} is called.

When GLFW\_AUTO\_POLL\_EVENTS is disabled, calling
\textbf{glfwSwapBuffers} will not result in a call to
\textbf{glfwPollEvents}. This can be useful if \textbf{glfwSwapBuffers}
needs to be called from within a callback function, since calling
\textbf{glfwPollEvents} from a callback function is not allowed.
\end{mysamepage}


\bigskip\begin{mysamepage}\hypertarget{lnk:keyrepeat}{}
\textbf{GLFW\_KEY\_REPEAT}\\
When GLFW\_KEY\_REPEAT is enabled, the key and character callback
functions are called repeatedly when a key is held down long enough
(according to the system key repeat configuration).

When GLFW\_KEY\_REPEAT is disabled, the key and character callback
functions are only called once when a key is pressed (and once when it is
released).
\end{mysamepage}


\bigskip\begin{mysamepage}\hypertarget{lnk:mousecursor}{}
\textbf{GLFW\_MOUSE\_CURSOR}\\
When GLFW\_MOUSE\_CURSOR is enabled, the mouse cursor is visible, and
mouse coordinates are relative to the upper left corner of the client area
of the \GLFW\ window. The coordinates are limited to the client area of
the window.

When GLFW\_MOUSE\_CURSOR is disabled, the mouse cursor is invisible, and
mouse coordinates are not limited to the drawing area of the window. It is
as if the mouse coordinates are recieved directly from the mouse, without
being restricted or manipulated by the windowing system.
\end{mysamepage}


\bigskip\begin{mysamepage}\hypertarget{lnk:stickykeys}{}
\textbf{GLFW\_STICKY\_KEYS}\\
When GLFW\_STICKY\_KEYS is enabled, keys which are pressed will not be
released until they are physically released and checked with
\textbf{glfwGetKey}. This behavior makes it possible to catch keys that
were pressed and then released again between two calls to
\textbf{glfwPollEvents} or \textbf{glfwSwapBuffers}, which would otherwise
have been reported as released. Care should be taken when using this mode,
since keys that are not checked with \textbf{glfwGetKey} will never be
released. Note also that enabling GLFW\_STICKY\_KEYS does not affect the
behavior of the keyboard callback functionality.

When GLFW\_STICKY\_KEYS is disabled, the status of a key that is reported
by \textbf{glfwGetKey} is always the physical state of the key. Disabling
GLFW\_STICKY\_KEYS also clears the sticky information for all keys.
\end{mysamepage}


\bigskip\begin{mysamepage}\hypertarget{lnk:stickymousebuttons}{}
\textbf{GLFW\_STICKY\_MOUSE\_BUTTONS}\\
When GLFW\_STICKY\_MOUSE\_BUTTONS is enabled, mouse buttons that are
pressed will not be released until they are physically released and
checked with \textbf{glfwGetMouseButton}. This behavior makes it
possible to catch mouse buttons which were pressed and then released again
between two calls to \textbf{glfwPollEvents} or \textbf{glfwSwapBuffers},
which would otherwise have been reported as released. Care should be taken
when using this mode, since mouse buttons that are not checked with
\textbf{glfwGetMouseButton} will never be released. Note also that
enabling GLFW\_STICKY\_MOUSE\_BUTTONS does not affect the behavior of the
mouse button callback functionality.

When GLFW\_STICKY\_MOUSE\_BUTTONS is disabled, the status of a mouse
button that is reported by \textbf{glfwGetMouseButton} is always the
physical state of the mouse button. Disabling GLFW\_STICKY\_MOUSE\_BUTTONS
also clears the sticky information for all mouse buttons.
\end{mysamepage}


\bigskip\begin{mysamepage}\hypertarget{lnk:systemkeys}{}
\textbf{GLFW\_SYSTEM\_KEYS}\\
When GLFW\_SYSTEM\_KEYS is enabled, pressing standard system key
combinations, such as \texttt{ALT+TAB} under Windows, will give the normal
behavior. Note that when \texttt{ALT+TAB} is issued under Windows in this
mode so that the \GLFW\ application is deselected when \GLFW\ is operating
in fullscreen mode, the \GLFW\ application window will be minimized and
the video mode will be set to the original desktop mode. When the \GLFW\
application is re-selected, the video mode will be set to the \GLFW\ video
mode again.

When GLFW\_SYSTEM\_KEYS is disabled, pressing standard system key
combinations will have no effect, since those key combinations are blocked
by \GLFW . This mode can be useful in situations when the \GLFW\ program
must not be interrupted (normally for games in fullscreen mode).
\end{mysamepage}


%-------------------------------------------------------------------------
\subsection{glfwGetNumberOfProcessors}

\textbf{C language syntax}
\begin{lstlisting}
int glfwGetNumberOfProcessors( void )
\end{lstlisting}

\begin{refparameters}
none
\end{refparameters}

\begin{refreturn}
The function returns the number of active processors in the system.
\end{refreturn}

\begin{refdescription}
The function determines the number of active processors in the system.
\end{refdescription}

\begin{refnotes}
Systems with several logical processors per physical processor, also
known as SMT (Symmetric Multi Threading) processors, will report the
number of logical processors.
\end{refnotes}


\end{document}
